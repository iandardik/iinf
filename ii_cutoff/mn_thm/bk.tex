\documentclass[12pt]{article}
\usepackage{anysize}
\marginsize{1.2cm}{1.4cm}{.4cm}{1cm}

\usepackage[normalem]{ulem}
\usepackage{amsmath}
\usepackage{amsfonts}
\usepackage{hyperref}
\usepackage{amsthm}

\newtheorem{lemma}{Lemma}

\title{The M-N Theorem}
\author{Ian Dardik}
\date{\today}


\begin{document}

\maketitle

\section{Introduction}
This is an experiment in determining the ``cutoff" based on the number of elements that are quantified in a property and a protocol's transition function.

\section{1 prop, 1 tr}
Suppose we have a transition function $T = (I,\delta)$ with template $T(P)$, i.e. $T$ has a single sort $P$.  Suppose the following property holds when $|P|=1$:
$$(\forall p \in P, \phi(p)) \land (\exists p \in P, \delta(p)) \rightarrow (\forall p \in P, \phi(p))'$$

Where $\phi(p)$ and $\delta(p)$ are quantifier-free predicates that can only refer to the given parameter in $P$.  Then when $|P|=2$ (without loss of generality):
\begin{align*}
  &(\forall p \in P, \phi(p)) \land (\exists p \in P, \delta(p))\\
  \iff &(\phi(1) \land \phi(2)) \land \delta(1)\\
  \iff &(\phi(1) \land \delta(1)) \land (\phi(2) \land \delta(1))\\
  \rightarrow\text{ } &\phi(1)' \land \phi(2)'\\
  \iff &(\forall p \in P, \phi(p))'
\end{align*}

Where the last step follows from the fact that $\delta(1)$ cannot refer to $2$, and hence cannot affect it.

\section{2 prop, 1 tr}
Suppose we have a transition function $T = (I,\delta)$ with template $T(P)$, i.e. $T$ has a single sort $P$.  Suppose the following property holds when $|P|=1$:
$$(\forall p,q \in P, \phi(p,q)) \land (\exists p \in P, \delta(p)) \rightarrow (\forall p,q \in P, \phi(p,q))'$$

It is not the case that this property must hold when $|P|=2$; the property $P = \forall p,q, (p \neq q) \rightarrow I$ is a clear counterexample.  Instead suppose that the property holds when $|P|=1$ \textit{and} $|P|=2$.  Then when $|P|=3$ (and without loss of generality):
\begin{align*}
  &(\forall p,q \in P, \phi(p)) \land (\exists p \in P, \delta(p))\\
  \iff &(\phi(1,1) \land \phi(2,2) \land \phi(3,3) \land \phi(1,2) \land \phi(2,1) \land \phi(2,3) \land \phi(3,2) \land \phi(1,3) \land \phi(3,1)) \land \delta(1)\\
  \iff &(\phi(1,1) \land \delta(1)) \land (\phi(2,2) \land \delta(1)) ...\\
  \rightarrow\text{ } &\phi(1,1)' \land \phi(2,2)' \land \phi(3,3)' \land \phi(1,2)' \land \phi(2,1)' \land \phi(2,3)' \land \phi(3,2)' \land \phi(1,3)' \land \phi(3,1)'\\
  \iff &(\forall p,q \in P, \phi(p,q))'
\end{align*}

\end{document}
