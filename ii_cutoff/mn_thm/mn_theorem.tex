\documentclass[12pt]{article}
\usepackage{anysize}
\marginsize{1.2cm}{1.4cm}{.4cm}{1cm}

\usepackage[normalem]{ulem}
\usepackage{amsmath}
\usepackage{amsfonts}
\usepackage{hyperref}
\usepackage{amsthm}
\usepackage{url}
\usepackage{hyperref}

\theoremstyle{definition}
\newtheorem{assumption}{Assumption}
\newtheorem{lemma}{Lemma}
\newtheorem{corollary}{Corollary}
\newtheorem{theorem}{Theorem}
\newtheorem{definition}{Definition}
\newtheorem{example}{Example}

\theoremstyle{remark}
\newtheorem{remark}{Remark}

\newcommand{\msp}{\text{ }}
\newcommand{\fips}{\text{FIPS}}
\newcommand{\perm}{\genfrac{}{}{0pt}{}}

\title{A Cutoff Rule For Parameterized Distributed Protocols in Prenex Normal Form}
\author{Ian Dardik}
\date{\today}


\begin{document}

\maketitle

\section{Introduction}
In this note, we consider the verification problem of a transition system $T=(I,\Delta)$ parameterized by a single sort $P$ of identical elements.  We assume that a candidate inductive invariant $\Phi$ (which implies our key safety property) is given, and that both $\Delta$ and $\Phi$ are in Prenex Normal Form (PNF).  We adopt the convention of \cite{goel2021symmetry} where $T(P)$ is the template of $T$, and $T(|P|)$ is a finite instantiation.  We also will consider the prime (') symbol to be an operator that can be recursively applied to a formula, only affecting (sticking to) state variables.

In this note, we will build several lemmas that lead to an interesting result: $\Phi(P)$ is an inductive invariant for $T(P)$ iff $\Phi(m+n)$ is an inductive invariant for $T(m+n)$, where $m$ is the number of variables that $\Phi$ quantifies over and $n$ is the number of variables that $\Delta$ quantifies over.  This result is useful for the verification problem laid out above because it reduces the burden to model checking the single finite instance $T(m+n)$.  Essentially, $m+n$ is a cutoff instance size for proving that our inductive invariant holds.


\section{Finitely Instantiated Properties (FIPs)}
In this section we introduce the FIP, a key tool for proving the M-N Theorem.

\subsection{FIP Basics}

\begin{definition}[FIP]
  Let $M_{\Phi}$ and $M_{\Delta}$ be the respective matrices of $\Phi$ and $\Delta$.  A Finitely Instantiated Property (FIP) of $[\Phi\land\Delta](|P|)$ is a formula $(M_{\Phi} \land M_{\Delta})[v_i \mapsto e_i]$, where each free variable $v_i$ has been substituted for a concrete element $e_i \in P$.  
\end{definition}

\begin{example}
  Let $\Phi = \forall p,q \in P, M_{\Phi}(p,q)$ and $\Delta = \exists p \in P, M_{\Delta}(p)$.  Then $M_{\Phi}(1,3) \land M_{\Delta}(2)$ and $M_{\Phi}(1,1) \land M_{\Delta}(1)$ are both FIPs of $[\Phi\land\Delta](3)$ (i.e. for the case when $P=\{1,2,3\}$).  
  \qed
\end{example}

\begin{remark}
  \label{rmk:fip-syntax}
  We will often write a FIPs of $[\Phi\land\Delta](|P|)$ in the abstract as $\phi\land\delta$.  We specifically choose ``$\phi\land\delta$" to show the syntactic correspondence of $\phi$ to $\Phi$ and $\delta$ to $\Delta$.  In other words, if $M_{\Phi}$ is the matrix of $\Phi$ and $M_{\Delta}$ is the matrix of $\Delta$, then it is always the case that $\phi = M_{\Phi}[v_i \mapsto e_i]$ and $\delta = M_{\Delta}[v_i \mapsto e_i]$.
\end{remark}

\begin{example}
  Let $\phi\land\delta = M_{\Phi}(1,3) \land M_{\Delta}(2)$ be a FIP of $[\Phi\land\Delta](3)$.  By Remark \ref{rmk:fip-syntax}, $\phi = M_{\Phi}(1,3)$ and $\delta = M_{\Delta}(2)$.
  \qed
\end{example}


\subsection{Protocol Assumptions}

Now that we have formally defined FIPs, we can describe our requirement that ``the elements of sort $P$ are identical" in precise terms.

\begin{assumption}[Sort Elements Are Identical]
  \label{asmp:perm}
  Let $\phi\land\delta$ be an arbitrary FIP of $[\Phi\land\Delta](|P|)$ and $g : P \to P$ be a bijective function.  Then we assume:
  $$\phi \leftrightarrow \phi[P \mapsto g(P)]$$
  \begin{center}
    and
  \end{center}
  $$\phi\land\delta \leftrightarrow (\phi\land\delta)[P \mapsto g(P)]$$
\end{assumption}

\begin{remark}
  In the context of Assumption \ref{asmp:perm}, $(\phi\land\delta)[P \mapsto g(P)]$ is said to be a \textit{permutation} of $\phi\land\delta$, and:
  $$(\phi\land\delta)[P \mapsto g(P)] = g(\phi\land\delta)$$
\end{remark}

\begin{example}
  Let $\Phi = \forall p \in P, M_{\Phi}(p)$ and $\Delta = \exists p,q \in P, M_{\Delta}(p,q)$.  Consider $M_{\Phi}(2) \land M_{\Delta}(1,3)$ and $M_{\Phi}(3) \land M_{\Delta}(1,2)$ which are both FIPs of $[\Phi\land\Delta](3)$.  Let $g$ be the permutation:
  $$g = \left(\perm{1 \msp 2 \msp 3}{1 \msp 3 \msp 2}\right)$$
  Then these two FIPs are permutations of each other because
  $$g(M_{\Phi}(2) \land M_{\Delta}(1,3)) = M_{\Phi}(3) \land M_{\Delta}(1,2)$$
  \begin{center}
    and
  \end{center}
  $$M_{\Phi}(2) \land M_{\Delta}(1,3) = g(M_{\Phi}(3) \land M_{\Delta}(1,2))$$
  \qed
\end{example}

\noindent Because each sort element is identical, we can make the following assumption without loss of generality:

\begin{assumption}
  \label{asmp:sort}
  For any sort $P$, $P = \{1,...,|P|\}$.  
\end{assumption}

\noindent This leads to the trivial result:

\begin{lemma}
  \label{lem:monot-card}
  Let $P$ and $Q$ be sorts, then $|P| \leq |Q| \leftrightarrow P \subseteq Q$.  
\end{lemma}
\begin{proof}
  Suppose $|P| \leq |Q|$.  Then by Assumption \ref{asmp:sort}:
  $$P = \{1,...,|P|\} \subseteq \{1,...,|Q|\} = Q$$
  Now suppose that $P \subseteq Q$.  Then by Assumption \ref{asmp:sort}:
  $$\{1,...,|P|\} = P \subseteq Q = \{1,...,|Q|\}$$
  Hence it follows that $|P| \leq |Q|$.
\end{proof}


\subsection{FIP Equivalency}

In this section we introduce the notion of \textit{equivalency} for FIPs and prove some useful lemmas.

\begin{definition}[Equivalent]
  Let $\phi_1\land\delta_1$ and $\phi_2\land\delta_2$ be FIPs, then $\phi_1\land\delta_1 \equiv \phi_2\land\delta_2$ iff $\phi_2\land\delta_2$ is a permutation of $\phi_1\land\delta_1$.  In this case, $\phi_1\land\delta_1$ and $\phi_2\land\delta_2$ are said to be \textit{equivalent}.
\end{definition}

\begin{example}
  Let $F_1 = M_{\Phi}(1,2) \land M_{\Delta}(1)$, $F_2 = M_{\Phi}(2,3) \land M_{\Delta}(2)$ and $F_3 = M_{\Phi}(2,2) \land M_{\Delta}(2)$ be FIPs of $[\Phi\land\Delta](3)$.  Let $g = (1 \msp 2 \msp 3)$ be a permutation (using cycle notation) on $P$.  Then $F_1 \equiv F_2$ because $g(F_1) = F_2$, however, $F_3$ is a permutation of neither $F_1$ nor $F_2$ and hence is not equivalent to either.
  \qed
\end{example}

The notion of equivalency is important because it partitions a formula $\Phi \land \Delta$ into distinct classes of FIPs.  In the example above, $F_1$ and $F_2$ describe the same class of properties/actions because each element of $P$ is interchangeable for one another.  We now present several basic lemmas about FIPs.

\begin{lemma}[Equivalency Is Commutative]
  \label{lem:fip-eq-comm}
  Let $\phi_1\land\delta_1$ and $\phi_2\land\delta_2$ be FIPs for $[\Phi\land\Delta](|P|)$, then
  $$(\phi_1\land\delta_1 \equiv \phi_2\land\delta_2) \leftrightarrow (\phi_2\land\delta_2 \equiv \phi_1\land\delta_1)$$
\end{lemma}
\begin{proof}
  Suppose that $\phi_1\land\delta_1 \equiv \phi_2\land\delta_2$.  Then there exists a permutation $g$ such that $g(\phi_1\land\delta_1) = \phi_2\land\delta_2$.  Permutations are bijective and hence are invertible.  Then
  $$\phi_1\land\delta_1 = g^{-1}(g(\phi_1\land\delta_1)) = g^{-1}(\phi_2\land\delta_2)$$
  But $g^{-1}$ is a permutation itself, and hence $\phi_1\land\delta_1$ is a permutation of $\phi_2\land\delta_2$.

  We omit the proof in the other direction because the argument is nearly identical.
\end{proof}

\begin{lemma}
  \label{lem:and-decomp}
  Let $\phi_1\land\delta_1$ and $\phi_2\land\delta_2$ both be FIPs of $[\Phi\land\Delta](|P|)$.  Then:
  $$((\phi_1\land\delta_1) \equiv (\phi_2\land\delta_2)) \rightarrow ((\phi_1 \equiv \phi_2) \land (\delta_1 \equiv \delta_2))$$
\end{lemma}
\begin{proof}
  Suppose $(\phi_1\land\delta_1) \equiv (\phi_2\land\delta_2)$.  Then there exists a permutation $g$ such that $g(\phi_1\land\delta_1) = \phi_2\land\delta_2$.  But:
  $$g(\phi_1) \land g(\delta_1) = g(\phi_1\land\delta_1) = \phi_2\land\delta_2$$
  From Remark \ref{rmk:fip-syntax}, we see that $\phi_1$ and $\phi_2$ are identical up to a finite instantiation, and hence $g(\phi_1) = \phi_2$; a similar argument shows that $g(\delta_1) = \delta_2$.
\end{proof}

\begin{lemma}
  \label{lem:prime-bijec}
  Let $\phi_1$ and $\phi_2$ be quantifier-free properties parameterized by the sort $P$.  Then:
  $$(\phi_1 \equiv \phi_2) \leftrightarrow (\phi_1' \equiv \phi_2')$$
\end{lemma}
\begin{proof}
  Suppose that $\phi_1 \equiv \phi_2$.  Then there exists a permutation $g$ such that $g(\phi_1) = \phi_2$.  Recall that the prime operator only affects state variables, and not $P$ or its elements; on the other hand, $g$ can only affect the elements of $P$.  Thus it is the case that $g(\phi_1)' = g(\phi_1')$.  We now see:
  $$\phi_2' = g(\phi_1)' = g(\phi_1')$$
  Showing that $\phi_1' \equiv \phi_2'$ by definition.  We omit the proof in the other direction since it is nearly identical.
\end{proof}

\begin{lemma}
  \label{lem:phi-eqiv-bijec}
  Let $\phi_1$ and $\phi_2$ be quantifier-free properties parameterized by the sort $P$.  Suppose that $\phi_1 \equiv \phi_2$.  Then:
  $$\phi_1 \leftrightarrow \phi_2$$
\end{lemma}
\begin{proof}
  Because $\phi_1 \equiv \phi_2$, there exists a permutation $g$ such that $g(\phi_1) = \phi_2$.  However $g$ is bijective, and hence by Assumption \ref{asmp:perm}:
  $$\phi_1 \leftrightarrow (\phi_1)[P \mapsto g(P)] = \phi_2$$
\end{proof}

\begin{lemma}
  \label{lem:fip-eqiv-bijec}
  Let $\phi_1\land\delta_1$ and $\phi_2\land\delta_2$ both be FIPs for $[\Phi\land\Delta](|P|)$.  Suppose that $\phi_1\land\delta_1 \equiv \phi_2\land\delta_2$.  Then:
  $$(\phi_1 \land \delta_1) \leftrightarrow (\phi_2 \land \delta_2)$$
\end{lemma}
\begin{proof}
  Because $\phi_1\land\delta_1 \equiv \phi_2\land\delta_2$, there exists a permutation $g$ such that $g(\phi_1\land\delta_1) = \phi_2\land\delta_2$.  However $g$ is bijective, and hence by Assumption \ref{asmp:perm}:
  $$\phi_1\land\delta_1 \leftrightarrow (\phi_1\land\delta_1)[P \mapsto g(P)] = \phi_2\land\delta_2$$
\end{proof}

\subsection{The FIPS Operator}

In this section we introduce the $\fips$ operator:
\begin{definition}
  Let $\Phi$ and $\Delta$ be PNF properties with respective matrices $\phi$ and $\delta$.  Suppose that $\Phi$ quantifies over $m \in \mathbb{N}$ variables while $\Delta$ quantifies over $n \in \mathbb{N}$ variables.  Then:
  $$\fips(\Phi \land \Delta, |P|) := \{\phi(v_1,...,v_m) \land \delta(w_1,...,w_n) | v_1,...,v_m,w_1,...,w_n \in P\}$$
\end{definition}

The $\fips$ operator simply contains every possible FIP for a given formula $[\Phi\land\Delta](|P|)$.  Next, we prove an intuitive result:

\begin{lemma}
  \label{lem:fips-subset}
  $\fips(\Phi \land \Delta, |P|) \subseteq \fips(\Phi \land \Delta, |Q|) \leftrightarrow |P| \leq |Q|$
\end{lemma}
\begin{proof}
  We begin by showing that $|P| \leq |Q| \rightarrow \fips(\Phi \land \Delta, |P|) \subseteq \fips(\Phi \land \Delta, |Q|)$.  Suppose $|P| \leq |Q|$, and then it follows that $P \subseteq Q$ by Lemma \ref{lem:monot-card}.  Then:
  \begin{align*}
    \fips(\Phi \land \Delta, |P|) = &\{\phi(v_1,...,v_m) \land \delta(w_1,...,w_n) | v_1,...,v_m,w_1,...,w_n \in P\}\\
    \subseteq &\{\phi(v_1,...,v_m) \land \delta(w_1,...,w_n) | v_1,...,v_m,w_1,...,w_n \in P\} \msp \cup\\
              &\msp \{\phi(v_1,...,v_m) \land \delta(w_1,...,w_n) | v_1,...,v_m,w_1,...,w_n \in Q \setminus P\}\\
    = &\{\phi(v_1,...,v_m) \land \delta(w_1,...,w_n) | v_1,...,v_m,w_1,...,w_n \in Q\}\\
    = &\fips(\Phi \land \Delta, |Q|)\\
  \end{align*}
  Next we show that $\fips(\Phi \land \Delta, |P|) \subseteq \fips(\Phi \land \Delta, |Q|) \rightarrow |P| \leq |Q|$.  Suppose that $\fips(\Phi \land \Delta, |P|) \subseteq \fips(\Phi \land \Delta, |Q|)$, then:
  \begin{align*}
    &\{\phi(v_1,...,v_m) \land \delta(w_1,...,w_n) | v_1,...,v_m,w_1,...,w_n \in P\}\\
    \subseteq &\{\phi(v_1,...,v_m) \land \delta(w_1,...,w_n) | v_1,...,v_m,w_1,...,w_n \in Q\}\\
  \end{align*}
  Which implies that $P \subseteq Q$, which in turn implies that $|P| \leq |Q|$ by Lemma \ref{lem:monot-card}.
\end{proof}


\section{M-N Theorem}

In this section we present the M-N Theorem.  First, we prove a key lemma that shows that FIPs saturate when $|P|$ grows large enough.

\begin{lemma}[FIP Saturation]
  \label{lem:fip-sat}
  Suppose that $\Phi$ quantifies over $m \in \mathbb{N}$ variables and $\Delta$ quantifies over $n \in \mathbb{N}$ variables.  Then every FIP of $[\Delta \land \Phi](k)$, for $k > m+n$, has an equivalent FIP in $[\Delta \land \Phi](m+n)$.
\end{lemma}
\begin{proof}
  Let $k = |P| = m + n + z$ where $z \in \mathbb{Z}_{>0}$, and then $P = \{1,...,k\}$ by Assumption \ref{asmp:sort}.  Let $\phi\land\delta$ be an arbitrary FIP of $[\Phi\land\Delta](k)$.  Then, because $[\Phi\land\Delta](k)$ quantifies over exactly $m+n$ variables, there must be at least $z$ elements in $P$ that do not appear in $\phi\land\delta$.  Let $u \leq m+n$ be the \textit{number} of elements of $P$ that are used in $\phi \land \delta$, and let $\{e_1,...,e_u\} \subseteq P$ be the \textit{set} of elements that are used.  Consider the following permutation:
  $$g = \left(\perm{e_1 \msp ... \msp e_u}{1 \msp ... \msp u}\right)$$
  Notice that $g(\phi\land\delta)$ contains only the elements $1 ... u$.  Since $u \leq m+n$, it must be the case that $g(\phi\land\delta)$ is a FIP of $[\Phi \land \Delta](m+n)$.
\end{proof}

\begin{theorem}[M-N]
  Suppose that $\Phi$ quantifies over $m \in \mathbb{N}$ variables and $\Delta$ quantifies over $n \in \mathbb{N}$ variables.  Then $\Phi(P)$ is an inductive invariant for $T(P)$ iff it is an inductive invariant for the finite instantiation $T(m+n)$.
\end{theorem}
\begin{proof}
  It is clear that if $\Phi(P)$ is an inductive invariant, then it must be an inductive invariant for $T(m+n)$.  We prove the opposite direction in the remainder of the proof.

  First, consider the case when the finite instantiation is less than or equal to $m+n$.  Suppose that $[\Phi\land\Delta](m+n) \rightarrow \Phi(m+n)'$.  Let $k \leq m+n$, then we must show that $[\Phi\land\Delta](k) \rightarrow \Phi(k)'$.  Consider an arbitrary FIP $\phi\land\delta$ of $[\Phi\land\Delta](k)$.  By Lemma \ref{lem:fips-subset}, $\phi\land\delta$ is also a FIP of $[\Phi\land\Delta](m+n)$, and therefore it follows that $\phi'$ holds.

  We will now focus on the case when the finite instantiation is larger than $m+n$.  Suppose that $[\Phi\land\Delta](m+n) \rightarrow \Phi(m+n)'$.  Let $k > m+n$, then we must show that $[\Phi\land\Delta](k) \rightarrow \Phi(k)'$.  Consider an arbitrary FIP $\phi\land\delta$ of $[\Phi\land\Delta](k)$.  By Lemma \ref{lem:fip-sat}, there exists a FIP $\phi_2\land\delta_2$ of $[\Phi\land\Delta](m+n)$ such that $\phi_2\land\delta_2 \equiv \phi\land\delta$.  By Lemma \ref{lem:fip-eqiv-bijec}, we see that $\phi_2\land\delta_2$ holds because $\phi\land\delta$ holds.  However, because $\phi_2\land\delta_2$ holds, and we know that $[\Phi\land\Delta](m+n) \rightarrow \Phi(m+n)'$, it follows that $\phi_2'$ holds.  Now Lemma \ref{lem:and-decomp} implies that $\phi \equiv \phi_2$, and Lemma \ref{lem:prime-bijec} implies that $\phi' \equiv \phi_2'$, so we have $\phi_2'$ and $\phi' \equiv \phi_2'$.  Thus by Lemma \ref{lem:phi-eqiv-bijec} and Lemma \ref{lem:fip-eq-comm}, we can conclude that $\phi'$ holds.
\end{proof}


\section{Case Studies}
In this section we visit several (more coming soon) distributed protocols that are parameterized by a single sort and adhere to the property of Assumption \ref{asmp:perm}.

\subsection{Peterson's Mutex Protocol}
Peterson's Mutex Protocol can be encoded with a transition function $\Delta$ in PNF that quantifies over two variables.  A sample inductive invariant candidate is given in \cite{ian-peterson} that quantifies of two variables and works for $|P|=2$:
\begin{verbatim}
    Phi == \A p,q \in ProcSet :
       /\ pc[p] \in {"a3","a4","cs"} => flag[p]
       /\ (p#q /\ pc[p] = "cs" /\ pc[q] = "a4") => turn = p
       /\ (p # q) => ~(pc[p] = "cs" /\ pc[q] = "cs")
\end{verbatim}
However, by the M-N Theorem, we must show that $\Phi$ is an inductive invariant for the case when $|P|=4$.  In fact, we easily see that $\Phi$ fails to be inductive in the case:
\begin{verbatim}
                      /\ turn = 1             /\ turn = 2
                      /\ pc[1] = "cs"         /\ pc[1] = "cs"
                      /\ pc[2] = "a4"   ->    /\ pc[2] = "a4"
                      /\ pc[3] = "a3"         /\ pc[3] = "a4"
                      /\ a3(3,2)
\end{verbatim}
This example uses states to describe the counterexample, but we can also describe it using the FIP $M_{\Phi}(1,2) \land M_{\Delta}(3,2)$ from $[\Phi\land\Delta](4)$.  When this FIP is true, both $M_{\Phi}(1,2)$ and $M_{\Phi}(1,3)$ fail to hold in the next state, showing that $M_{\Phi}(1,2)$--and hence $\Phi$--is not inductive.

This example shows how a FIP describes a specific relationship between $\Phi$ and $\Delta$; in this case the specific relationship leads to a counterexample.  It is important to note that it is only possible to describe this particular counterexample using a FIP with a minimum of three elements in $P$, which is precisely why we do not detect the counter example in Peterson's Protocol when $|P|=2$.

It is also worthwhile to note that we could derive the same counterexample using an equivalent FIP, say $M_{\Phi}(3,2) \land M_{\Delta}(1,2)$.  This shows how FIP equivalency partitions a formula $\Phi\land\Delta$ into classes of specific relationships that a transition system can exhibit.


\bibliographystyle{plain}
\bibliography{refs}

\end{document}
