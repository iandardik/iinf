\documentclass[12pt]{article}
\usepackage{anysize}
\marginsize{1.2cm}{1.4cm}{.4cm}{1cm}

\usepackage[normalem]{ulem}
\usepackage{amsmath}
\usepackage{amsfonts}
\usepackage{hyperref}
\usepackage{amsthm}
\usepackage{url}
\usepackage{hyperref}

\theoremstyle{definition}
\newtheorem{assumption}{Assumption}
\newtheorem{lemma}{Lemma}
\newtheorem{corollary}{Corollary}
\newtheorem{theorem}{Theorem}
\newtheorem{definition}{Definition}
\newtheorem{example}{Example}

\theoremstyle{remark}
\newtheorem{remark}{Remark}

\newcommand{\msp}{\text{ }}
\newcommand{\st}{\text{ }|\text{ }}
\newcommand{\states}{\text{States}}
\newcommand{\gr}{\text{Gr}}
\newcommand{\fips}{\text{FIPS}}
\newcommand{\perm}{\genfrac{}{}{0pt}{}}

\title{A Cutoff Rule For A Special Class Of Parameterized Distributed Protocols}
\author{Ian Dardik}
\date{\today}


\begin{document}

\maketitle

\section{Introduction}
In this note, we consider the verification problem of a transition system $T=(I,\Delta)$ where $I$ is the initial constraint, $\Delta$ is the transition relation, and the system is parameterized by a single sort $P$ of identical elements (We make the notion of ``identical" precise in Assumption \ref{asmp:ident} below).  We assume that we are given a candidate inductive invariant $\Phi$ which implies our key safety property.  $\Phi$ is restricted to be in Prenex Normal Form (PNF) with only universal quantifiers, while $\Delta$ is restricted to be in PNF with only existential quantifiers.  We adopt the convention of \cite{goel2021symmetry} where $T(P)$ is the template of $T$, and $T(|P|)$ is a finite instantiation.

In this note, we will build several lemmas that lead to an interesting result: let $m$ be the number of variables that $\Phi$ quantifies over and $n$ be the number of variables that $\Delta$ quantifies over, then if $\Phi(m+n)$ is an inductive invariant, $\Phi(k)$ is also an inductive invariant for all $k>m+n$.  We will refer to this as the M-N Theorem in this note.  This result is useful because it reduces the vefification problem on $T$ to model checking a finite number of instances $T(1),T(2),....,T(m+n)$.  Essentially, $m+n$ is a cutoff instance size for proving that our inductive invariant holds.

Note: I think it is likely that if $\Phi(m+n)$ is an inductive invariant, then it is \textit{also} the case for $\Phi(k)$ for all $k<m+n$, but I left this out of this note for the time being to focus on the $k>m+n$ case.



\section{Preliminaries}
In this section we cover several preliminary items that we use to prove the M-N Theorem.

\subsection{Without Loss Of Generality}
We will assume that the parameter $P = \{1,2,...,|P|\}$.  This assumption comes without loss of generality because each member of $P$ is assumed to be identical.  This assumption also implies that for any two finite sort instances $P$ and $Q$, $|P| \leq |Q| \leftrightarrow P \subseteq Q$.

\subsection{Assumptions}
This section contains the list of assumptions for the transition system we work with.  In other words, these assumptions are the requirements for the M-N Theorem to hold.

\begin{assumption}[$P$ Has Identical Elements]
  \label{asmp:ident}
  Let $f$ be a ground formula and let $\pi : P \to P$ be a bijective function, i.e. a permutation on $P$.  Then we assume:
  $$f \leftrightarrow \pi(f)$$
\end{assumption}

\subsection{Definitions}
\begin{definition}[States]
  Let $k \in \mathbb{N}$, then:
  $$\states(k) := \{s \st s \text{ is a state of } T(k)\}$$
  In this note we consider a state $s \in \states(k)$ to be a quantifier-free formula: a conjunction of constraints that describe a single state in $T(k)$.
\end{definition}

\begin{definition}[Satisfaction]
  Let $f$ and $g$ be formulas in First Order Logic.  Then we say $f \models g$ iff $f \rightarrow g$.  Alternatively, $f$ satisfies $g$ iff $f$ is stronger than $g$.
\end{definition}
\begin{example}
  Consider the transition system $T(P)$ with two state variables, $x \in (P \to \mathbb{N})$ and $y \in \mathbb{Z}$.  Let $|P|=2$, then $P=\{1,2\}$.  Let $s := (x[1]=6 \land x[2]=0 \land y=-22)$ be a state in the transition system.  Let $F := \forall p,q \in P, x[p] \neq x[q]$ and $f := (x[1] \neq x[2])$.  Then $f$ is a ground formula of $F(2)$, $F(2) \models f$, $s \models F(2)$, and $s \models f$.
\end{example}

\begin{definition}[Ground Formulas]
  Let $F$ be a quantified formula and $k \in \mathbb{N}$.
  $$\gr(F,k) := \{f \st (f \text{ is a ground formula of } F(k)) \land (F(k) \models f)\}$$
\end{definition}

\begin{example}
  $\gr([\forall p,q \in P, p=q],2) = \{(1=1),(1=2),(2=1),(2=2)\}$\\
  Note: we sometimes use square braces to wrap formulas when it looks better than parentheses.\\
  Notice that $\gr([\forall p,q \in P, p=q],2)$ contains elements that are false.  This indicates that the statement $[\forall p,q \in P, p=q](2)$ is not valid.
\end{example}
\begin{example}
  Let $\text{sv}$ be a state variable, then:
  $$\gr((\forall p,q \in P, p \neq q \rightarrow \text{sv[p]} \neq \text{sv[q]}),3) = \{(1 \neq 1 \rightarrow \text{sv[1]} \neq \text{sv[1]}),(1 \neq 2 \rightarrow \text{sv[1]} \neq \text{sv[2]}),...\}$$
\end{example}



\section{Helper Lemmas}

\begin{lemma}
  \label{lem:pnf-ground}
  Let $k \in \mathbb{N}$, $s \in \states(k)$, and $F$ be a universally quantified formula.  Then:
  $$(s \models F(k)) \leftrightarrow (\forall f \in \gr(F,k), s \models f)$$
\end{lemma}
\begin{proof}
  Suppose that $s \models F(k)$.  For an arbitrary formula $f \in \gr(F,k)$, $F(k) \models f$ and hence we see that $s \rightarrow F(k) \land F(k) \rightarrow f$.  It follows that $s \models f$.

  Now suppose that $\forall f \in \gr(F,k), s \models f$.  Suppose, for the sake of contradiction, that $s \not\models F(k)$.  Then it must be the case that $s \land \neg F(k)$.  We know that $F$ is unversally quantified, so let $F(k) := \forall \hat{x}, \phi(\hat{x})$ where $\hat{x}$ is the vector of variables that we quantify over.  Then it must be the case that $\exists \hat{x}, \neg \phi(\hat{x})$, but $\phi(\hat{x}) \in \gr(F,k)$.  However this contradicts our original assumption, and hence the result is proved.
\end{proof}

\begin{lemma}
  \label{lem:lt-sat}
  Let $k \in \mathbb{N}$, and $s \in \states(k)$ such that $s \models \Phi(k)$.  Then for $j \leq k$, it is also the case that $s \models \Phi(j)$.
\end{lemma}
\begin{proof}
  Let $k$ and $j \leq k$ be given and suppose that $s \models \Phi(k)$.  By Lemma \ref{lem:pnf-ground}, $\forall f \in \gr(F,k), s \models \Phi(k)$.  Now observe that $\gr(\Phi,j) \subseteq \gr(\Phi,k)$ due to the fact that $\Phi$ is a universally quantified PNF formula.  Thus it is also the case that $\forall f \in \gr(F,j), s \models \Phi(j)$, and then the result follows from Lemma \ref{lem:pnf-ground}.
  %Side note: the observation $\gr(\Phi,j) \subseteq \gr(\Phi,k)$ also shows that $\Phi(k) \rightarrow \Phi(j)$ by the Curry-Howard Isomorphishm.
\end{proof}

\begin{lemma}
  \label{lem:state-sat-perm}
  Let $s$ be a state, $f$ be a ground formula, and $\pi$ be a permutation.  Then:
  $$(s \models f) \leftrightarrow (\pi(s) \models \pi(f))$$
\end{lemma}
\begin{proof}
  Suppose that $s \models f$, and hence $s \rightarrow f$.  By Assumption \ref{asmp:ident}, $s \leftrightarrow \pi(s)$ and $f \leftrightarrow \pi(f)$, and the result follows immediately.

  Now suppose that $\pi(s) \models \pi(f)$.  $\pi$ is a bijection--and hence invertible--thus $\pi^{-1}$ is a permutation as well.  By Assumption \ref{asmp:ident}, $\pi(s) \leftrightarrow \pi^{-1}(\pi(s)) = s$ and $\pi(f) \leftrightarrow \pi^{-1}(\pi(f)) = f$.  The result follows immediately.
\end{proof}

\begin{lemma}
  \label{lem:state-perm}
  Let $k \in \mathbb{N}$ and $s$ be a state such that $s \models \Phi(k)$.  If $\pi$ is a permutation then it is also the case that $\pi(s) \models \Phi(k)$.
\end{lemma}
\begin{proof}
  Suppose that $s \models \Phi(k)$.  Then by Lemma \ref{lem:pnf-ground}, $\forall f \in \gr(\Phi,k), s \models f$.  But Assumption \ref{asmp:ident} shows that $s \leftrightarrow \pi(s)$ and hence $\forall f \in \gr(\Phi,k), \pi(s) \models f$ which gives us our result by Lemma \ref{lem:pnf-ground}.
\end{proof}



\section{The M-N Theorem}
\begin{theorem}[M-N]
  Suppose that $\Phi$ is in PNF with only universal quantifiers, while $\Delta$ is in PNF with only existential quantifiers.  Let $m$ be the number of variables that $\Phi$ quantifies over and $n$ be the number of variables that $\Delta$ quantifies over.  If $\Phi(m+n)$ is an inductive invariant, then $\Phi(k)$ is also an inductive invariant for any $k>m+n$.
\end{theorem}
\begin{proof}
  Assume that $[\Phi\land\Delta \rightarrow \Phi'](m+n)$ is valid.  Let $k>m+n$ be given and $s \in \states(k)$ such that $s \models \Phi(k)$.  Let $\delta$ be a single arbitrary transition such that $\delta \models \Delta(k)$.  Finally, let $f' \in \gr(\Phi',k)$ be arbitrary, then, by Lemma \ref{lem:pnf-ground}, it suffices to show that $(s \land \delta) \models f'$.

  Next, we will construct a permutation $\pi$ as follows: let $x_1,...,x_j$ be the distinct elements of $P$ used in $\delta$ and $f'$.  We know that $j \leq m+n$ because $\Delta$ quantifies over $n$ variables while $\Phi$ quantifies over $m$ variables.  Let:
  $$\pi := \left( \perm{x_1 \msp x_2 \msp ... \msp x_j}{1 \msp\msp 2 \msp\msp ... \msp\msp j} \right)$$
  Notice that the formulas $\pi(\delta)$ and $\pi(s)$ now only contain the elements $1,...,j$ and, in particular, $\pi(\delta) \models \Delta(m+n)$ and $\pi(f') \in \gr(\Phi',m+n)$, i.e. $\pi(f') \models \Phi'(m+n)$.  By Lemma \ref{lem:lt-sat} we see that $s \models \Phi(m+n)$, and furthermore $\pi(s) \models \Phi(m+n)$ by Lemma \ref{lem:state-perm}.  Thus $\pi(s \land \delta) \models [\Phi\land\Delta](m+n)$ which implies $\pi(s \land \delta) \models \Phi'(m+n)$ by our initial assumption.  In particular, $\pi(s \land \delta) \models \pi(f')$ by Lemma \ref{lem:pnf-ground}, and therefore $s \land \delta \models f'$ by Lemma \ref{lem:state-sat-perm}.
\end{proof}



\section{Case Studies}
In this section we visit several (more coming soon) distributed protocols that are parameterized by a single sort and satisfy Assumption \ref{asmp:ident}.

\subsection{Peterson's Mutex Protocol}
Peterson's Mutex Protocol can be encoded with a transition relation $\Delta$ in PNF that quantifies over two variables.  A sample inductive invariant candidate is given in \cite{ian-peterson} that quantifies of two variables and works for $|P|=2$:
\begin{verbatim}
    Phi == \A p,q \in ProcSet :
       /\ pc[p] \in {"a3","a4","cs"} => flag[p]
       /\ (p#q /\ pc[p] = "cs" /\ pc[q] = "a4") => turn = p
       /\ (p # q) => ~(pc[p] = "cs" /\ pc[q] = "cs")
\end{verbatim}
However, by the M-N Theorem, we must show that $\Phi$ is an inductive invariant for the cases when $|P|=1,...,4$.  In fact, we easily see that $\Phi(3)$ fails to be inductive in the following counter example:
\begin{verbatim}
                      /\ turn = 1             /\ turn = 2
                      /\ pc[1] = "cs"         /\ pc[1] = "cs"
                      /\ pc[2] = "a4"   ->    /\ pc[2] = "a4"
                      /\ pc[3] = "a3"         /\ pc[3] = "a4"
                      /\ a3(3,2)
\end{verbatim}


\bibliographystyle{plain}
\bibliography{refs}

\end{document}
