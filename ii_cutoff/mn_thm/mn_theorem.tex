\documentclass[12pt]{article}
\usepackage{anysize}
\marginsize{1.2cm}{1.4cm}{.4cm}{1cm}

\usepackage[normalem]{ulem}
\usepackage{amsmath}
\usepackage{amsfonts}
\usepackage{hyperref}
\usepackage{amsthm}

\newtheorem{lemma}{Lemma}
\newtheorem{theorem}{Theorem}
\newtheorem{definition}{Definition}
\newcommand{\msp}{\text{ }}

\title{The M-N Theorem}
\author{Ian Dardik}
\date{\today}


\begin{document}

\maketitle

\section{Introduction}
I begin with some preliminaries before introducing the M-N Theorem.


\section{Preliminaries}
Throughout this note we will consider a transition system $T=(I,\Delta)$ parameterized by a single sort $P$ with identical elements (i.e. each element is interchangeable for another).  $\Delta$ is the transition relation for $T$ and we stipulate that it is in Prenex Normal Form (PNF).  $\Phi$ is an inductive invariant candidate that is also in PNF, and our goal is to determine whether or not $\Phi$ is an inductive invariant for $T$.

Because $\Phi$ and $\Delta$ are in PNF, will also can refer directly to the matrices of these formulas as $\phi$ and $\delta$ respectively; i.e. $\phi$ and $\delta$ are propositional logic formulas parameterized by the variables that are quantified over in $\Phi$ and $\Delta$ respectively.

\begin{definition}
  Let $\Phi$ and $\Delta$ be of two PNF formulas and let $\phi$ and $\delta$ be their respective matrices.  Assume that $\Phi$ quantifies over $m \in \mathbb{N}$ variables while $\Delta$ quantifies over $n \in \mathbb{N}$ variables.  Then a Finitely Instantiated Property (FIP) of $\Phi \land \Delta$ is a formula $(\phi \land \delta) [v_i \mapsto j]$, where each free variable $v_i$ has been substituted for a concrete element $j \in P$.  
\end{definition}

\noindent \textbf{Example:}\\
Let $\Phi = \forall p,q \in P, \phi(p,q)$ and $\Delta = \exists p \in P, \delta(p)$.  Then if $P= \{1,2,3\}$ is a finite instantiation of $T$, the formulas $\phi(1,3) \land \delta(2)$ and $\phi(1,1) \land \delta(1)$ are both FIPs of $\Phi \land \Delta$.

\begin{definition}
  Two FIPs $F_1 = \phi_1 \land \delta_1$ and $F_2 = \phi_2 \land \delta_2$ are equivalent iff $F_1$ is a permutation of $F_2$.
\end{definition}

\noindent \textbf{Example:}\\
Let $P=\{1,2,3\}$, $F_1 = \phi_1(1,2) \land \delta(1)$, $F_2 = \phi_2(2,3) \land \delta(2)$ and $F_3 = \phi(2,2) \land \delta(2)$.  Then $F_1 \equiv F_2$ because $F_1 \msp (1 \msp 2 \msp 3) = F_2$ (using cycle notation).  However $F_3$ is a permutation of neither $F_1$ nor $F_2$ and hence is not equivalent to both.\\

The notion of equivalency is important because it partitions a FIP into distinct classes of action types.  In the example above, $F_1$ and $F_2$ describe the same class of property and action because each element of $P$ is interchangeable for one another.  This leads us to the following lemma that is rather intuitive:

\begin{lemma}
  Let $\phi_1\land\delta_1$ and $\phi_2\land\delta_2$ be FIPs for $\Phi_1\land\Delta_1$ and $\Phi_2\land\Delta_2$ respectively.  Suppose that $\phi_1\land\delta_1 \equiv \phi_2\land\delta_2$, i.e. there exists a cycle $C$ such that $(\phi_1\land\delta_1) \msp C = \phi_2\land\delta_2$.  Then
  $$((\phi_1 \land \delta_1) \msp C \rightarrow \phi_1') \leftrightarrow (\phi_2 \land \delta_2 \rightarrow \phi_2')$$
\end{lemma}
\begin{proof}
  This result follows immediately from the fact that each element of $P$ is interchangeable for one another.
\end{proof}


\section{Intuition}
We will build intuition by proving the M-N Theorem for small examples.  Coming soon.


\section{M-N Theorem}

\begin{lemma}
  Let $\Phi$ and $\Delta$ be formulas in PNF, where $\Phi$ quantifies over $m \in \mathbb{N}$ variables and $\Delta$ quantifies over $n \in \mathbb{N}$ variables.  Then any FIP of $\Delta \land \Phi$ that appears when $|P| > m+n$ also appears when $|P| = m+n$.
\end{lemma}
\begin{proof}
  Let $|P| = m + n + z$ where $z \in \mathbb{Z}_{>0}$.  Then, because $\phi$ and $\delta$ are parameterized by exactly $m+n$ variables, there must be at least $z$ unused variables (very similar to the Pigeonhole Principle).  Let $P = \{v_i\}_{i=1}^{m+n+z}$ where each $v_i$ is a variable, let $u \leq m+n$ be the \textit{number} of variables that are used in $\phi \land \delta$, and finally let $\{v_{i_k}\}_{k=1}^{u}$ be the \textit{set} of variables that are used.  Consider the permuation using the following cycle notation: $C = (v_{i_1} v_1)...(v_{i_u} v_u)$.  It is clear that $(\phi \land \delta) \msp C \equiv (\phi \land \delta)$, but notice that $(\phi \land \delta) \msp C$ only uses variables $v_1 ... v_u$.  Since $u \leq m+n$, it must be the case that $(\phi \land \delta) \msp C$ is a FIP of $\Phi \land \Delta$ when $|P| = m+n$.
\end{proof}

\begin{theorem}
  Let $\Phi$ and $\Delta$ be formulas in PNF, where $\Phi$ quantifies over $m \in \mathbb{N}$ variables and $\Delta$ quantifies over $n \in \mathbb{N}$ variables.  Then $\Phi$ is an inductive invariant for $T(P)$ iff it is an inductive invariant for the finite instantiation $T(m+n)$.
\end{theorem}
\begin{proof}
  It is clear that if $\Phi$ is an inductive invariant, then it must be an inductive invariant for $T(m+n)$.  We prove the opposite direction in the remainder of the proof.

  We will skip the case when the finite instantiation is less than $m+n$ and focus when it is larger for now.

  Suppose that $\Phi(m+n) \land \Delta(m+n) \rightarrow \Phi(m+n)'$.  Let $k > m+n$, then we must show that $\Phi(k) \land \Delta(k) \rightarrow \Phi(k)'$.  Consider the FIP when $|P|=k$: $\phi(1 ... m) \land \delta(1 ... n)$.  By Lemma 2, we know that this FIP exists in $T(m+n)$, and hence we have a cycle $R$ and a permutation $(\phi(1 ... m) \land \delta(1 ... n) \msp R)$ that only contains the variables $v_1 ... v_{m+n}$.  By Lemma 1, $(\phi(1 ... m) \land \delta(1 ... n) \msp R) \rightarrow (\phi(1 ... m)' \msp R)$, which is equivalent to $\phi(1 ... m)'$ by definition.
\end{proof}

\end{document}
