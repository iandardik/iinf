\documentclass[12pt]{article}
\usepackage{anysize}
\marginsize{1.2cm}{1.4cm}{.4cm}{1cm}

\usepackage[normalem]{ulem}
\usepackage{amsmath}
\usepackage{amsfonts}
\usepackage{hyperref}
\usepackage{amsthm}
\usepackage{url}
\usepackage{hyperref}

\theoremstyle{definition}
\newtheorem{assumption}{Assumption}
\newtheorem{lemma}{Lemma}
\newtheorem{corollary}{Corollary}
\newtheorem{theorem}{Theorem}
\newtheorem{definition}{Definition}
\newtheorem{example}{Example}

\theoremstyle{remark}
\newtheorem{remark}{Remark}

\newcommand{\msp}{\text{ }}
\newcommand{\st}{\text{ }|\text{ }}
\newcommand{\states}{\text{States}}
\newcommand{\gr}{\text{Gr}}
\newcommand{\elems}{\text{Elems}}
\newcommand{\PQF}{\text{PQF}}
\newcommand{\replace}{\text{Replace}}
\newcommand{\perm}{\genfrac{}{}{0pt}{}}

\title{A Cutoff Rule For A Special Class Of Parameterized Distributed Protocols}
\author{Ian Dardik}
\date{\today}


\begin{document}

\maketitle

\section{Introduction}
In this note, we consider the verification problem of a transition system $T=(I,\Delta)$ where $I$ is the initial constraint, $\Delta$ is the transition relation, and the system is parameterized by a single sort $E$ of indistinguishable elements (We make the notion of ``indistinguishable" precise in Assumption \ref{asmp:indist} below).  

To begin, we will introduce notation for the template and finite instances of a transition system.  We adopt the convention of \cite{goel2021symmetry} where $T(P)$ is the template of $T$ and $T(|P|)$ is a finite instance.  We can also refer to the template or a finite instance of a quantified formula $F$ and the sort $P$.  For example, suppose $F$ is in Prenex Normal Form (PNF) and univerally quantifies over $j$ variables, i.e. $F$ can be written as:
$$F := \forall x_1,...,x_j \in E, \phi(x_1,...,x_j)$$
where $\phi$ is a non-quantified statement whose only free variables are $x_1,...,x_j$.  Then $F(k)$ is identical to the formula $F$, except $E$ is replaced by $E(k) \subseteq E$, where $|E(k)|=k$.  Throughout this note we will let $E(k)=\{e_1,...,e_k\}$.  Thus we see:
$$F(k) = \forall x_1,...,x_j \in E(k), \phi(x_1,...,x_j)$$

In this note, we are concerned with the specific scenario in which we are given a candidate inductive invariant $\Phi$, and the finite instances $\Phi(1),...,\Phi(k)$ have been proved to be inductive invariants for $T(1),...,T(k)$; we want to know whether $\Phi$ is an inductive invariant for $T$.  We are specifically concerned with the case in which both $\Delta$ and $\Phi$ are written in PNF and $\Phi$ is restricted to universal quantification.

Throughout this note, we will build several lemmas that lead to an interesting result: let $m$ be the number of variables that $\Phi$ quantifies over and $n$ be the number of variables that $\Delta$ quantifies over; if we suppose that $\Phi(m+n)$ is an inductive invariant for $T(m+n)$, then $\Phi(k)$ is also an inductive invariant for $T(k)$ for all $k>m+n$.  We will refer to this as the M-N Theorem in this note.  This result is useful because it reduces the verification problem on $T$ to model checking a finite number of instances $T(1),...,T(m+n)$.  Essentially, $m+n$ is a cutoff instance size for proving that our inductive invariant holds.

Note: I think it is likely that if $\Phi(m+n)$ is an inductive invariant, then it is \textit{also} the case for $\Phi(k)$ for all $k<m+n$, but I left this out of this note for the time being to focus on the $k>m+n$ case.



\section{Preliminaries}
In this section we introduce several assumptions, definitions, and notation that we will use to prove the M-N Theorem.


\subsection{Restricted Sorted Logic}
In this section we introduce a restricted form of Many Sorted Logic that we will refer to as Restricted Sorted Logic (RSL).  This logic includes a single sort $S$, and we define its syntax based on the grammar for FOL defined in \cite{ben-ari}.

\begin{definition}
  Let $\mathbf{P}$ be a countable set of predicate symbols and $\mathbf{V}$ be a countable set of variables.  Then a formula in RSL is generated by the following grammar:
  \begin{align*}
    argument &::= x \hfill \text{ for any } x \in \mathbf{V}\\
    argument\_list &::= argument\\
    argument\_list &::= argument,argument\_list\\
    formula &::= p(argument\_list) \hfill \text{ for any } n\text{-ary } p \in \mathbf{P}, n \geq 0\\
    formula &::= \forall x \in S \msp formula \hfill \text{ for any } x \in \mathbf{V}\\
    formula &::= \exists x \in S \msp formula \hfill \text{ for any } x \in \mathbf{V}\\
  \end{align*}
  Where predicates are $n$-ary functions that return boolean values and $S$ is the sort given in an interpretation.  Notice that uninterpreted quantified formulas appear in prenext normal form (PNF).  We will implicitly assume that any quantified formula is a closed formula for the remainder of this note.
\end{definition}

\begin{definition}
  An \textit{interpretation} $\mathbf{I}$ is a pair $(S, R)$.  $D$ is a non-empty sort of indistinguishable constants, and $R$ is a finite set of relations that serve as interpretations to each predicate symbol.  The relations of $R$ may be parameterized by the elements of $S$, but they may not have knowledge of $S$ itself; in particular, the relations may not quantify over $S$ nor refer to its cardinality.
\end{definition}

\begin{example}
  Let $U := \forall x \in S, \exists y \in S, p_1(x,y)$ and $\mathbf{I} = (\mathbb{Z}, \{<\})$.  Then $U$ is an (uninterpreted) quantified formula and $I$ is an interpretation of $U$.  $U$ interpreted with $I$ can be described by the formula:
  $$\forall x \in \mathbb{Z}, \exists y \in \mathbb{Z}, x < y$$
  We will often abuse the notation and simply write:
  $$\forall x, \exists y \in \mathbb{Z}, x < y$$
\end{example}


\subsection{Transition System Basics}

Let $T=(I,\Delta)$ be our transition system with sort $E$ and inductive invariant candidate $\Phi$.  Throughout this note, we assume that the fomulas $I$, $\Delta$, and $\Phi$ are written in RSL and share the interpretation $\mathbf{I} = (E, R)$.  $R$ is defined by a set of formulas on the state variables of $T$, and parameterized by the elements of $E$.

For a given $k \in \mathbb{N}$, we write $\mathbf{I}(k) = (E(k), R)$ to denote the interpretation that is shared between the finite instances $I(k)$, $\Delta(k)$, and $\Phi(k)$.

\begin{example}
  Consider the transition system $T$ with two state variables, $A \in (E \to \mathbb{N})$ and $b \in \mathbb{Z}$ where
  \begin{align*}
    I &:= \forall x \in E, p_1(x)\\
    \Delta &:= \exists x \in E, p_2(x)\\
    \Phi &:= \forall x,y \in E, p_3(x,y)\\
  \end{align*}
  This particular transition system has the interpretation $(E,\{r_1,r_2,r_3\})$, where each relation $r_i$ corresponds to each predicate $p_i$.  The three relations can be described concisely by the following three formulas:
  \begin{align*}
    r_1(x) &:= (A[x] = 1) \land (b = 0)\\
    r_2(x) &:= (A[x] = 1) \land (A'[x] = 2) \land (b' = b+1)\\
    r_3(x,y) &:= (A[x] + A[y] < 5) \land (b \geq 0)\\
  \end{align*}
  Notice how the formulas of each relation does \textit{not} refer to $E$.  In summary, our interpreted transition system can be described using the trio of formulas:
  \begin{align*}
    I &:= \forall x \in E, (A[x] = 1) \land (b = 0)\\
    \Delta &:= \exists x \in E, (A[x] = 1) \land (A'[x] = 2) \land (b' = b+1)\\
    \Phi &:= \forall x,y \in E, (A[x] + A[y] < 5) \land (b \geq 0)\\
  \end{align*}
\end{example}

\begin{definition}
  A formula $\Phi$ is an inductive invariant for $T$ iff $\forall k \in \mathbb{N}$, $\Phi(k)$ is an inductive invariant for $T(k)$.  $\Phi(k)$ is an inductive invariant for $T(k)$ iff $I(k) \rightarrow \Phi(k)$ and $[\Phi \land \Delta \rightarrow \Phi'](k)$ are both valid formulas.
\end{definition}


\subsection{Ground Formulas}

\begin{definition}
  Let $S$ be a sort.  Then a \textit{ground formula} is generated by the following grammar:
  \begin{align*}
    argument &::= e \hfill \text{ for any } e \in S\\
    argument\_list &::= argument\\
    argument\_list &::= argument,argument\_list\\
    ground\_formula &::= p(argument\_list) \hfill \text{ for any } n\text{-ary } p \in \mathbf{P}, n \geq 0\\
  \end{align*}
  %We will let $G := \{g \st g \text{ is generated by } gr\_formula\}$ be the universe of all ground formulas.
\end{definition}

\begin{definition}[States]
  Let $k \in \mathbb{N}$, then:
  $$\states(k) := \{s \st s \text{ is a state of } T(k)\}$$
  In this note we consider a ``state" $s \in \states(k)$ to be a ground formula.  More specifically--under a given interpretation for $T$--$s$ is a conjunction of constraints that describe a single state in $T(k)$.
\end{definition}

\begin{definition}
  Let $F$ be an RSL formula and $\rho : \mathbf{V} \to E$ be a function.  Then we define $\replace(F,\rho)$ recursively
  \begin{align*}
    \replace(x,\rho) &:= \rho(x) \hfill \text{ for any } x \in \mathbf{V}\\
    \replace((argument,argument\_list),\rho) &:= \replace(argument,rho), \replace(argument\_list,\rho)\\
    \replace(p(argument\_list),\rho) &:= p(\replace(argument\_list,\rho)) \hfill \text{ for any } n\text{-ary } p \in \mathbf{R}, n \geq 0\\
  \end{align*}
\end{definition}

\begin{definition}[Ground Instance of \textit{F(k)}]
  Let $F$ be a quantified PNF formula and $k \in \mathbb{N}$.  Then $g$ is a \textit{ground instance} of $F(k)$ iff there exists a mapping $\rho : \mathbf{V} \to E(k)$ and an unquantified formula $f$ such that:
  $$g = \replace(f,\rho) \text{ and } F \in \PQF(f)$$
  In other words, $g$ is a ground formula that is identical in structure to $F$ without quantifiers, and with all variables of $F(k)$ replaced by members of $E(k)$.
\end{definition}

\begin{example}
  Consider the transition system $T$ with two state variables, $x \in (P \to \mathbb{N})$ and $y \in \mathbb{Z}$.  Let $s := (x[1]=6 \land x[2]=0 \land y=-22)$ be a state in the transition system.  Let $F := \forall p,q \in P, x[p] \neq x[q]$ and $f := (x[1] \neq x[2])$.

  Then $F(2) = \forall p,q \in P(2), x[p] \neq x[q]$.  Furthermore, $f$ is a ground instance of $F(2)$, $F(2) \models f$, $s \models F(2)$, and $s \models f$.
\end{example}

\begin{definition}[Gr]
  Let $F$ be a quantified formula and $k \in \mathbb{N}$.  Then:
  $$\gr(F,k) := \{f \st f \text{ is a ground instance of } F(k)\}$$
\end{definition}

\begin{example}
  $\gr([\forall p,q \in P, p=q],2) = \{(1=1),(1=2),(2=1),(2=2)\}$\\
  Note: we sometimes use square braces to wrap formulas when it looks better than parentheses.\\
  Notice that $\gr([\forall p,q \in P, p=q],2)$ contains elements that are false.  This indicates that the statement $[\forall p,q \in P, p=q](2)$ is not valid.
\end{example}
\begin{example}
  Let $\text{sv}$ be a state variable, then:
  $$\gr((\forall p,q \in P, p \neq q \rightarrow \text{sv[p]} \neq \text{sv[q]}),3) = \{(1 \neq 1 \rightarrow \text{sv[1]} \neq \text{sv[1]}),(1 \neq 2 \rightarrow \text{sv[1]} \neq \text{sv[2]}),...\}$$
\end{example}

\begin{definition}[Elems]
  Suppose that $F$ is a quantified formula, $k \in \mathbb{N}$, and $f \in \gr(F,k)$.  Then:
  $$\elems(f) := \{e \st e \in P(k) \land e \text{ occurs in } f\}$$

  TODO make this definition better.
\end{definition}


\subsection{Permutation Transformations}

\begin{definition}[Permutation Transformation]
  Let $k \in \mathbb{N}$, $\pi : P(k) \to P(k)$ be a permutation on $P(k)$, and $G$ be the set of all possible formulas.  Then $M_\pi : G \to G$ is the \textit{permutation transformation} on $\pi$, a syntactic transformation that replaces each element from $P(k)$ in a formula with its permuted value.
\end{definition}

\begin{example}
  Let $\pi$ be the following permutation:
  $$\pi := \left( \perm{1 \msp 2 \msp 3}{2 \msp 3 \msp 1} \right)$$
  Let $\text{sv}$ be a state variable, then:
  $$M_\pi(3 \neq 1 \rightarrow \text{sv[3]} \neq \text{sv[1]}) = (1 \neq 2 \rightarrow \text{sv[1]} \neq \text{sv[2]})$$
\end{example}

\subsection{Indistinguishable Elements}
We have loosely stipulated that $T$ must have ``indistinguishable" elements.  In this section, we make this assumption precise.

\begin{assumption}[$P$ Has Indistinguishable Elements]
  \label{asmp:indist}
  %Let $f$ and $g$ be two formulas and let $k \in \mathbb{N}$ be given.  If $\pi : P(k) \to P(k)$ is a permutation on $P(k)$, then:
  %$$(f \models g) \leftrightarrow (M_\pi(f) \models M_\pi(g))$$
  Let $j,k \in \mathbb{N}$ such that $j \geq k$ and $F$ be a quantified sentence in PNF.  Let $s \in \states(j)$ such that $s \models F(k)$.  If $\pi$ is a permutation then it is also the case that $M_\pi(s) \models F(k)$.
\end{assumption}



\section{Helper Lemmas}

\begin{lemma}
  \label{lem:pnf-ground}
  Let $j,k \in \mathbb{N}$ such that $j \geq k$, $s \in \states(j)$, and $F$ be a universally quantified formula.  Then:
  $$(s \models F(k)) \leftrightarrow (\forall f \in \gr(F,k), s \models f)$$
\end{lemma}
\begin{proof}
  Suppose that $s \models F(k)$.  For an arbitrary formula $f \in \gr(F,k)$, $F(k) \models f$ and hence we see that $s \rightarrow F(k) \land F(k) \rightarrow f$.  It follows that $s \models f$.

  Now suppose that $\forall f \in \gr(F,k), s \models f$.  Suppose, for the sake of contradiction, that $s \not\models F(k)$.  Then it must be the case that $s \land \neg F(k)$.  We know that $F$ is unversally quantified, so let $F(k) := \forall x_1,...,x_m \in P, \phi(x_1,...,x_m)$ where $m \geq 1$.  Then, because $\neg F(k)$ holds, it must be the case that $\exists x_1,...,x_m \in P, \neg \phi(x_1,...,x_m)$.  However, $\phi(x_1,...,x_m) \in \gr(F,k)$ which, by our original assumption, implies $\neg s$.  Hence we have both $s$ and $\neg s$ and we have reached a contradiction.
\end{proof}

\begin{lemma}[Gr Members Sat]
  \label{lem:gr-sat}
  Let $F$ be a formula and $k \in \mathbb{N}$ be given.  Then:
  $$\forall g \in \gr(F,k), F(k) \models g$$
  **THIS IS ONLY TRUE FOR UNIVERSALLY QUANTIFIED $F$**
\end{lemma}
%\begin{proof}
  %Let $F(k)$ be an arbitrary RSL formula:
  %$$F(k) = Q_1 \msp x_1,...,Q_m \msp x_m \in E(k), p(x_1,...,x_m)$$
  %where each $Q_i \in \{\forall,\exists\}$.  Suppose that $g \in \gr(F,k)$, then $g = \text{Unquant}(F(k)) = p(e_{x_1},...,e_{x_m})$ for some $e_{x_1},...,e_{x_m} \in E(k)$.  It is clearly the case that:
  %$$Q_1 \msp x_1,...,Q_m \msp x_m \in E(k), p(x_1,...,x_m) \rightarrow p(e_{x_1},...,e_{x_m})$$
%\end{proof}

\begin{lemma}[Gr Closed Under Permutation]
  \label{lem:gr-closed}
  Let $g$ be a ground formula, $F$ be a quantified formula, and $k \in \mathbb{N}$ be given.  Let $\pi : P(k) \to P(k)$ be a permutation, then:
  $$(g \in \gr(F,k)) \leftrightarrow (M_\pi(g) \in \gr(F,k))$$
\end{lemma}
\begin{proof}
  Suppose that $g \in \gr(F,k)$, then there exists a formula $f$ such that $g = f[\mathbf{V} \mapsto P(k)]$ and $F \in \PQF(f)$.  However:
  $$M_\pi(g) = M_\pi(f[\mathbf{V} \mapsto P(k)]) = f[\mathbf{V} \mapsto \pi(P(k))]$$

  The other direction is straightforward if we realize that $\pi^{-1}$ is also a permutation.
\end{proof}

\begin{lemma}[Minimum Gr]
  \label{lem:min-gr}
  Let $F$ be a formula, $f$ be a ground formula, and $k \in \mathbb{N}$ be given.  Suppose that $\elems(f) \subseteq P(j)$ where $j \leq k$, then:
  $$f \in \gr(F,k) \rightarrow f \in \gr(F,j)$$
\end{lemma}
\begin{proof}
  I will need a better definition for $\gr$ to prove this one.  For now, proof by obviousness.
\end{proof}

\begin{lemma}
  \label{lem:lt-sat}
  Let $k \in \mathbb{N}$, and $s \in \states(k)$ such that $s \models \Phi(k)$.  Then for $j \leq k$, it is also the case that $s \models \Phi(j)$.
\end{lemma}
\begin{proof}
  Let $k$ and $j \leq k$ be given and suppose that $s \models \Phi(k)$.  By Lemma \ref{lem:pnf-ground}, $\forall f \in \gr(F,k), s \models \Phi(k)$.  Now observe that $\gr(\Phi,j) \subseteq \gr(\Phi,k)$ due to the fact that $\Phi$ is a universally quantified PNF formula.  Thus it is also the case that $\forall f \in \gr(F,j), s \models \Phi(j)$, and then the result follows from Lemma \ref{lem:pnf-ground}.
  %Side note: the observation $\gr(\Phi,j) \subseteq \gr(\Phi,k)$ also shows that $\Phi(k) \rightarrow \Phi(j)$ by the Curry-Howard Isomorphishm.
\end{proof}



\section{The M-N Theorem}

In this section, we will establish initiation and consecution in two separate lemmas using similar techniques.  The M-N Theorem follows immediately from these two lemmas.

\begin{lemma}[M-N Initiation]
  Suppose that $\Phi(m)$ is an inductive invariant for $T(m)$, then $I(k) \rightarrow \Phi(k)$ for all $k>m$.
\end{lemma}
\begin{proof}
  Coming soon.
  %Because $\Phi(m)$ is an inductive invariant for $T(m)$, it is the case that $I(m) \rightarrow \Phi(m)$.  Now consider $I(k)$ where $k>m$.  Let $s \in \states(k)$ be arbitrary with the condition $s \models I(k)$, and now it suffices to prove that $s \models \Phi(k)$.  

  %Next, we will construct a permutation as follows: let $x_1,...,x_j$ be the distinct elements of $P(k)$ used in $\Phi(k)$.  We know that $j \leq m$ because $\Phi$ quantifies over $m$ variables.  Let:
  %$$\pi := \left( \perm{x_1 \msp x_2 \msp ... \msp x_j}{1 \msp\msp 2 \msp\msp ... \msp\msp j} \right)$$
  %Notice that $M_\pi(s)$ only contains the elements $1,...,j$.  

  %Next, by Lemma \ref{lem:lt-sat} (needs to be updated) and the fact that $s \models I(k)$, we know that $s \models I(m)$.  By Assumption \ref{asmp:indist}, we also know that $M_\pi(s) \models I(m)$.  Hence $M_\pi(s) \models \Phi(m)$.  By Assumption \ref{asmp:indist}--noting the fact that $\pi^{-1}$ is also a permutation of $P(k)$--we also see that $s \models \Phi(m)$.  
\end{proof}

\begin{lemma}[M-N Consecution]
  Suppose that $\Phi$ is in PNF with only universal quantifiers, while $\Delta$ is in PNF with only existential quantifiers.  Let $m$ be the number of variables that $\Phi$ quantifies over and $n$ be the number of variables that $\Delta$ quantifies over.  If $\Phi(m+n)$ is an inductive invariant, then $\Phi(k)$ is inductive for any $k>m+n$.
\end{lemma}
\begin{proof}
  Assume that $[\Phi\land\Delta \rightarrow \Phi'](m+n)$ is valid.  Let $k>m+n$ be given and $s \in \states(k)$ such that $s \models \Phi(k)$.  Let $\delta \in \gr(\Delta,k)$, i.e. $\delta$ is a ground ``transition".  Let $t \in \states(k)$ such that $t' \models (s \land \delta)$, that is, $t'$ is an arbitrary ``next" state of $s$.  Finally, let $f' \in \gr(\Phi',k)$ be arbitrary, then, by Lemma \ref{lem:pnf-ground} and the fact that $\Phi'$ is in PNF and universally quantified, it suffices to show that $t' \models f'$.

  Next, we will construct a permutation $\pi$ as follows: let $x_1,...,x_j$ be the distinct elements of $P(k)$ used in $\delta$ and $f'$.  We know that $j \leq m+n$ because $\Delta$ quantifies over $n$ variables while $\Phi$ quantifies over $m$ variables.  Then:
  $$\pi := \left( \perm{x_1 \msp x_2 \msp ... \msp x_j}{1 \msp\msp 2 \msp\msp ... \msp\msp j} \right)$$
  And hence by construction, $\elems(M_\pi(\delta)) \subseteq P(j)$ and $\elems(M_\pi(f')) \subseteq P(j)$.  First, we immediately see that $M_\pi(\delta) \in \gr(\Delta,k)$ and $M_\pi(f') \in \gr(\Phi',k)$ by Lemma \ref{lem:gr-closed}.  Next, we further notice:
  $$\elems(M_\pi(\delta)) \subseteq P(j) \subseteq P(m+n)$$
  \begin{center}
    and
  \end{center}
  $$\elems(M_\pi(f')) \subseteq P(j) \subseteq P(m+n)$$
  And thus by Lemma \ref{lem:min-gr}, we see that $M_\pi(\delta) \in \gr(\Delta,m+n)$ and $M_\pi(f') \in \gr(\Phi',m+n)$.

  Now because $s \models \Phi(k)$, we see that $s \models \Phi(m+n)$ by Lemma \ref{lem:lt-sat}, and furthermore $M_\pi(s) \models \Phi(m+n)$ by Assumption \ref{asmp:indist}.  Notice:
  $$M_\pi(t' \models (s \land \delta)) \leftrightarrow M_\pi(t' \rightarrow (s \land \delta)) \leftrightarrow (M_\pi(t') \rightarrow M_\pi(s \land \delta)) \leftrightarrow M_\pi(t') \models M_\pi(s \land \delta)$$
  Now:
  $$M_\pi(t') \models M_\pi(s \land \delta) = M_\pi(s) \land M_\pi(\delta) \models [\Phi(m+n) \land \Delta(m+n)] = [\Phi\land\Delta](m+n)$$
  Thus $M_\pi(t') \models [\Phi\land\Delta](m+n)$, which in turn implies $M_\pi(t') \models \Phi'(m+n)$ by our initial assumption.

  Informal: 
  Notice that $\elems(M_\pi(\delta)) \subseteq P(m+n)$, and hence the elements of the set $P(k) - P(m+n)$ have the same constraints in $t'$ as they do in $s$; this means $M_\pi(t') \models (\Phi(k) - \Phi(m+n))'$ (for an abuse of notation).  This further implies that $M_\pi(t') \models \Phi(k)$, and hence by Assumption \ref{asmp:indist}, it follows that $t' \models \Phi(k)'$.  In particular, by Lemma \ref{lem:pnf-ground}, $t' \models f'$.

  %In particular, by Lemma \ref{lem:gr-sat} and the fact that $M_\pi(f') \in \gr(\Phi',m+n)$, we see:
  %$$M_\pi(t') \models \Phi'(m+n) \land \Phi'(m+n) \models M_\pi(f')$$
  %And thus it is the case that $M_\pi(t') \models M_\pi(f')$.  Finally, by Assumption \ref{asmp:indist}, it follows that $t' \models f'$.
\end{proof}

Next we present the M-N Theorem:

\begin{theorem}[M-N]
  Suppose that $\Phi$ is in PNF with only universal quantifiers, while $\Delta$ is in PNF with only existential quantifiers.  Let $m$ be the number of variables that $\Phi$ quantifies over and $n$ be the number of variables that $\Delta$ quantifies over.  If $\Phi(m+n)$ is an inductive invariant, then $\Phi(k)$ is also an inductive invariant for any $k>m+n$.
\end{theorem}
\begin{proof}
  This follows immediately from the previous two lemmas.
\end{proof}



\section{Case Studies}
In this section we visit several (more coming soon) distributed protocols that are parameterized by a single sort and satisfy Assumption \ref{asmp:indist}.

\subsection{Peterson's Mutex Protocol}
Peterson's Mutex Protocol can be encoded with a transition relation $\Delta$ in PNF that quantifies over two variables.  A sample inductive invariant candidate is given in \cite{ian-peterson} that quantifies of two variables and works for $|P|=2$:
\begin{verbatim}
    Phi == \A p,q \in ProcSet :
       /\ pc[p] \in {"a3","a4","cs"} => flag[p]
       /\ (p#q /\ pc[p] = "cs" /\ pc[q] = "a4") => turn = p
       /\ (p # q) => ~(pc[p] = "cs" /\ pc[q] = "cs")
\end{verbatim}
However, by the M-N Theorem, we must show that $\Phi$ is an inductive invariant for the cases when $|P|=1,...,4$.  In fact, we easily see that $\Phi(3)$ fails to be inductive in the following counter example:
\begin{verbatim}
                      /\ turn = 1             /\ turn = 2
                      /\ pc[1] = "cs"         /\ pc[1] = "cs"
                      /\ pc[2] = "a4"   ->    /\ pc[2] = "a4"
                      /\ pc[3] = "a3"         /\ pc[3] = "a4"
                      /\ a3(3,2)
\end{verbatim}


\bibliographystyle{plain}
\bibliography{refs}

\end{document}
