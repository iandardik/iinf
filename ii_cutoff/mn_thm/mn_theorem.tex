\documentclass[12pt]{article}
\usepackage{anysize}
\marginsize{1.2cm}{1.4cm}{.4cm}{1cm}

\usepackage[normalem]{ulem}
\usepackage{amsmath}
\usepackage{amsfonts}
\usepackage{hyperref}
\usepackage{amsthm}
\usepackage{url}
\usepackage{hyperref}

\theoremstyle{definition}
\newtheorem{assumption}{Assumption}
\newtheorem{lemma}{Lemma}
\newtheorem{corollary}{Corollary}
\newtheorem{theorem}{Theorem}
\newtheorem{definition}{Definition}
\newtheorem{example}{Example}

\theoremstyle{remark}
\newtheorem{remark}{Remark}

\newcommand{\msp}{\text{ }}
\newcommand{\states}{\text{States}}
\newcommand{\gr}{\text{Gr}}
\newcommand{\fips}{\text{FIPS}}
\newcommand{\perm}{\genfrac{}{}{0pt}{}}

\title{A Cutoff Rule For A Special Class Of Parameterized Distributed Protocols}
\author{Ian Dardik}
\date{\today}


\begin{document}

\maketitle

\section{Introduction}
In this note, we consider the verification problem of a transition system $T=(I,\Delta)$ parameterized by a single sort $P$ of identical elements.  We assume that a candidate inductive invariant $\Phi$ (which implies our key safety property) is given.  $\Phi$ universally quantifies over one or more variables, $\Delta$ (the transition relation) exitentially quantifies over one or more variables, and and both $\Phi$ and $\Delta$ are in Prenex Normal Form (PNF).  We adopt the convention of \cite{goel2021symmetry} where $T(P)$ is the template of $T$, and $T(|P|)$ is a finite instantiation.  We also will consider the prime (') symbol to be an operator that can be recursively applied to a formula, only affecting (sticking to) state variables.

In this note, we will build several lemmas that lead to an interesting result: $\Phi(P)$ is an inductive invariant for $T(P)$ iff $\Phi(m+n)$ is an inductive invariant for $T(m+n)$, where $m$ is the number of variables that $\Phi$ quantifies over and $n$ is the number of variables that $\Delta$ quantifies over.  This result is useful for the verification problem laid out above because it reduces the burden to model checking the single finite instance $T(m+n)$.  Essentially, $m+n$ is a cutoff instance size for proving that our inductive invariant holds.



\section{Preliminaries}
In this section we cover several preliminary items that we use to prove the MN Theorem.

\subsection{Without Loss Of Generality}
We will assume that the parameter $P = \{1,...,|P|\}$.  This assumption comes without loss of generality because each member of $P$ is assumed to be identical.  We make the notion of ``identical" precise in Assumption \ref{asmp:ident}.

\subsection{Assumptions}
This section contains the list of assumptions for the transition system we work with.  In other words, these assumptions are the requirements for the M-N Theorem to hold.

\begin{assumption}[$P$ Has Identical Elements]
  \label{asmp:ident}
  Let $f$ be a ground formula and let $\pi : P \to P$ be a bijective function, i.e. a permutation on $P$.  Then we assume:
  $$f \leftrightarrow \pi(f)$$
\end{assumption}

\subsection{Definitions}
\begin{definition}[States]
  Let $k \in \mathbb{N}$, then:
  $$\states(k) := \{\text{all states when } |P|=k\}$$
  In this note we consider a state $s$ to be a formula: a conjunction of constraints that describe a single state in the transition system.
\end{definition}

\begin{definition}[Ground Formulas]
  Let $F$ be a quantified formula and $k \in \mathbb{N}$.
  $$\gr(F,k) := \{f | (f \text{ is a ground formula of } F(k)) \land (f \models F(k))\}$$
\end{definition}

\begin{example}
  $\gr((\forall p,q, p=q),2) := \{(1=1),(2=2)\}$
\end{example}
\begin{example}
  $\gr((\forall p,q, p \neq q),3) := \{(1 \neq 2),(1 \neq 3),(2 \neq 1),(2 \neq 3),(3 \neq 1),(3 \neq 2)\}$
\end{example}
\begin{remark}
  \label{rmk:pnf-ground}
  Notice that for any state $s \in \states(k)$ and quantified formula $F$:
  $$(s \models F(k)) \leftrightarrow (\forall f \in \gr(F,k), s \models f)$$
\end{remark}



\section{Helper Lemmas}

\begin{lemma}
  \label{lem:lt-sat}
  Let $k \in \mathbb{N}$, and $s \in \states(k)$ such that $s \models \Phi(k)$.  Then for $j \leq k$, it is also the case that $s \models \Phi(j)$.
\end{lemma}
\begin{proof}
  Let $j \leq k$ be given.  We will begin by observing that $\gr(\Phi,j) \subseteq \gr(\Phi,k)$ due to the fact that $\Phi$ is a universally quantified PNF formula.  The result then follows immediately from Remark \ref{rmk:pnf-ground}.
\end{proof}

% technically we need to assume that s is a dijunction of one or more state.  The proof is identical, but things may be more akward to state.
\begin{lemma}
  \label{lem:state-sat-perm}
  Let $s$ be a state, $f$ be a ground formula, and $\pi$ be a permutation.  Then:
  $$(s \models f) \leftrightarrow (\pi(s) \models \pi(f))$$
\end{lemma}
\begin{proof}
  Suppose that $s \models f$, which is syntactic sugar for $s \rightarrow f$ because $s$ and $f$ are both formulas.  By Assumption \ref{asmp:ident}, $s \leftrightarrow \pi(s)$ and $f \leftrightarrow \pi(f)$, and the result follows immediately.

  Now suppose that $\pi(s) \models \pi(f)$.  $\pi$ is a bijection--and hence invertible--thus $\pi^{-1}$ is a permutation as well.  By Assumption \ref{asmp:ident}, $\pi(s) \leftrightarrow \pi^{-1}(\pi(s)) = s$ and $\pi(f) \leftrightarrow \pi^{-1}(\pi(f)) = f$.  The result follows immediately.
\end{proof}

\begin{lemma}
  \label{lem:state-perm}
  Let $k \in \mathbb{N}$ and $s$ be a state such that $s \models \Phi(k)$.  If $\pi$ is a permutation then it is also the case that $\pi(s) \models \Phi(k)$.
\end{lemma}
\begin{proof}
  Suppose that $s \models \Phi(k)$.  Then by Remark \ref{rmk:pnf-ground}, $\forall f \in \gr(\Phi,k), s \models f$.  But Assumption \ref{asmp:ident} shows that $s \leftrightarrow \pi(s)$ and hence $\forall f \in \gr(\Phi,k), \pi(s) \models f$ which gives us our result by Remark \ref{rmk:pnf-ground}.
\end{proof}



\section{MN}
\begin{theorem}[M-N]
  Suppose that $\Phi$ is in PNF with only universal quantifiers, while $\Delta$ is in PNF with only existential quantifiers.  Let $m$ be the number of variables that $\Phi$ quantifies over and $n$ be the number of variables that $\Delta$ quantifies over.  If $\Phi(m+n)$ is an inductive invariant, then $\Phi(k)$ is also an inductive invariant for any $k>m+n$.
\end{theorem}
\begin{proof}
  Let $k>m+n$ be given and assume that $[\Phi\land\Delta \rightarrow \Phi'](m+n)$ is valid.  Let $s \in \states(k)$ such that $s \models \Phi(k)$, and let $\delta$ be an arbitrary transition such that $\delta \models \Delta(k)$.  Finally, let $f' \in \gr(\Phi',k)$ be arbitrary, then, by Remark \ref{rmk:pnf-ground}, it suffices to show that $(s \land \delta) \models f'$.

  Let $\pi$ be a permutation such that $\pi(\delta) \models \Delta(m+n)$ and $\pi(f') \in \gr(\Phi',m+n)$, i.e. $\pi(f') \models \Phi'(m+n)$.  We know that we can find such a $\pi$ because $\delta$ will contain at most $n$ distinct elements of $P$ and $f'$ will contain at most $m$ distinct elements of $P$.  Now by Lemma \ref{lem:lt-sat} we see that $s \models \Phi(m+n)$, and furthermore $\pi(s) \models \Phi(m+n)$ by Lemma \ref{lem:state-perm}.  Thus $\pi(s \land \delta) \models [\Phi\land\Delta](m+n)$ which implies $\pi(s \land \delta) \models \Phi'(m+n)$ by our initial assumption.  In particular, $\pi(s \land \delta) \models \pi(f')$ by Remark \ref{rmk:pnf-ground}, and therefore $s \land \delta \models f'$ by Lemma \ref{lem:state-sat-perm}.
\end{proof}



\section{Case Studies}
In this section we visit several (more coming soon) distributed protocols that are parameterized by a single sort and adhere to the property of Assumption \ref{asmp:ident}.

\subsection{Peterson's Mutex Protocol}
Peterson's Mutex Protocol can be encoded with a transition function $\Delta$ in PNF that quantifies over two variables.  A sample inductive invariant candidate is given in \cite{ian-peterson} that quantifies of two variables and works for $|P|=2$:
\begin{verbatim}
    Phi == \A p,q \in ProcSet :
       /\ pc[p] \in {"a3","a4","cs"} => flag[p]
       /\ (p#q /\ pc[p] = "cs" /\ pc[q] = "a4") => turn = p
       /\ (p # q) => ~(pc[p] = "cs" /\ pc[q] = "cs")
\end{verbatim}
However, by the M-N Theorem, we must show that $\Phi$ is an inductive invariant for the case when $|P|=4$.  In fact, we easily see that $\Phi$ fails to be inductive in the case:
\begin{verbatim}
                      /\ turn = 1             /\ turn = 2
                      /\ pc[1] = "cs"         /\ pc[1] = "cs"
                      /\ pc[2] = "a4"   ->    /\ pc[2] = "a4"
                      /\ pc[3] = "a3"         /\ pc[3] = "a4"
                      /\ a3(3,2)
\end{verbatim}
This example uses states to describe the counterexample, but we can also describe it using the FIP $M_{\Phi}(1,2) \land M_{\Delta}(3,2)$ from $[\Phi\land\Delta](4)$.  When this FIP is true, both $M_{\Phi}(1,2)$ and $M_{\Phi}(1,3)$ fail to hold in the next state, showing that $M_{\Phi}(1,2)$--and hence $\Phi$--is not inductive.

This example shows how a FIP describes a specific relationship between $\Phi$ and $\Delta$; in this case the specific relationship leads to a counterexample.  It is important to note that it is only possible to describe this particular counterexample using a FIP with a minimum of three elements in $P$, which is precisely why we do not detect the counter example in Peterson's Protocol when $|P|=2$.

It is also worthwhile to note that we could derive the same counterexample using an equivalent FIP, say $M_{\Phi}(3,2) \land M_{\Delta}(1,2)$.  This shows how FIP equivalency partitions a formula $\Phi\land\Delta$ into classes of specific relationships that a transition system can exhibit.


\bibliographystyle{plain}
\bibliography{refs}

\end{document}
