\documentclass[12pt]{article}
\usepackage{anysize}
\marginsize{1.2cm}{1.4cm}{.4cm}{1cm}

\usepackage[normalem]{ulem}
\usepackage{amsmath}
\usepackage{amsfonts}
\usepackage{hyperref}
\usepackage{amsthm}
\usepackage{url}
\usepackage{hyperref}

\theoremstyle{definition}
\newtheorem{assumption}{Assumption}
\newtheorem{lemma}{Lemma}
\newtheorem{corollary}{Corollary}
\newtheorem{theorem}{Theorem}
\newtheorem{definition}{Definition}
\newtheorem{example}{Example}

\theoremstyle{remark}
\newtheorem{remark}{Remark}

\newcommand{\msp}{\text{ }}
\newcommand{\st}{\text{ }|\text{ }}
\newcommand{\states}{\text{States}}
\newcommand{\gr}{\text{Gr}}
\newcommand{\elems}{\text{Elems}}
\newcommand{\perm}{\genfrac{}{}{0pt}{}}

\title{M-N In EPR}
\author{Ian Dardik}
\date{\today}


\begin{document}

\maketitle

%\section{M-N In EPR}
Suppose that $\Phi$ is universally quantified and in PNF while $\Delta$ is existentially quantified and in PNF.  We will assume a single sort $P$, and let $\Phi$ quantify over $m$ variables and $\Delta$ quantify over $n$ variables.  We will write $\Phi = \forall \hat{p}, \phi(\hat{p})$, $\Delta = \exists \hat{q}, \delta(\hat{q})$, and $\Phi' = \forall \hat{r}, \phi(\hat{r})'$, where $\phi$ and $\delta$ are non-quantified, uninterpreted, and contain no function symbols.  It is the case that $|\hat{p}| = |\hat{r}| = m$ and $|\hat{q}| = n$.  We use different quantifer variables for $\Phi$ and $\Phi'$ so the variables are already standardized apart.

Then we can prove the M-N Theorem for this special case quite quickly:

\begin{align*}
  & \msp \Phi \land \Delta \rightarrow \Phi'\\
  \leftrightarrow & \msp (\forall \hat{p}, \phi(\hat{p})) \land (\exists \hat{q}, \delta(\hat{q})) \rightarrow (\forall \hat{r}, \phi(\hat{r})')\\
  \leftrightarrow & \msp \neg(\forall \hat{p}, \phi(\hat{p})) \lor \neg(\exists \hat{q}, \delta(\hat{q})) \lor (\forall \hat{r}, \phi(\hat{r})')\\
  \leftrightarrow & \msp (\exists \hat{p}, \neg\phi(\hat{p})) \lor (\forall \hat{q}, \neg\delta(\hat{q})) \lor (\forall \hat{r}, \phi(\hat{r})')\\
  \leftrightarrow & \msp (\forall \hat{q}, \neg\delta(\hat{q})) \lor (\forall \hat{r}, \phi(\hat{r})') \lor (\exists \hat{p}, \neg\phi(\hat{p}))\\
  \leftrightarrow & \msp \forall \hat{q}, \forall \hat{r}, \exists \hat{p}, \neg\delta(\hat{q}) \lor \phi(\hat{r})' \lor \neg\phi(\hat{p})\\
\end{align*}

In other words, we want to know if:
$$\forall \hat{q}, \forall \hat{r}, \exists \hat{p}, \neg\delta(\hat{q}) \lor \phi(\hat{r})' \lor \neg\phi(\hat{p})$$
is valid.  We can turn this into a SAT question by negating the formula, which results in:
$$\exists \hat{q}, \exists \hat{r}, \forall \hat{p}, \neg\delta(\hat{q}) \lor \phi(\hat{r})' \lor \neg\phi(\hat{p})$$
This formula is in EPR, and according to Proposition 14.2 (Libkin, Elements of Finite Model Theory), if the formula is satisfiable then it has a model whose universe is size $|\hat{r}| + |\hat{q}| = m + n$.

%\bibliographystyle{plain}
%\bibliography{refs}

\end{document}
