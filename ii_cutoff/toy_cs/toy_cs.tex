\documentclass[12pt]{article}
\usepackage{anysize}
\marginsize{1.2cm}{1.4cm}{.4cm}{1cm}

\usepackage[normalem]{ulem}
\usepackage{amsmath}
\usepackage{amsfonts}
\usepackage{hyperref}
\usepackage{amsthm}

\newtheorem{lemma}{Lemma}

\title{Verification of ToyCS Using a Cutoff}
\author{Ian Dardik}
\date{\today}


\begin{document}

\maketitle

\section{Introduction}
We introduce the ToyCS protocol and present a key safety property.  We then prove protocol correctness using a cutoff proof.

\section{ToyCS}
ToyCS is encoded in TLA+ as follows:
\begin{verbatim}
  Init == cs = {}

  Next ==
      \E p \in ProcSet :
          /\ cs = {}
          /\ cs' = cs \cup {p}

  TypeOK == cs \in SUBSET ProcSet

  Safety == \A p,q \in cs : p = q
\end{verbatim}

The variable \textit{cs} represents the critical section, while \textit{Safety} is effectively mutual exclusion.  ToyCS is trivially simple by design.  $Safety$ is in fact an inductive invariant itself, and happens to be exactly equal to $Reach$, the set of all reachable states.

\section{Verification}

\subsection{Why Use a Cutoff Proof?}
\label{ssec:why}
ToyCS and its key safety property are trivial; standard techniques such as model checking and the invariant method can easily be leveraged to verify ToyCS.  We will demonstrate correctness using the invariant method--specifically using a cutoff proof to prove consecution--in hopes that eventually we will discover a more general cutoff proof technique that can be automated.

\subsection{Cutoff Proofs}
There are many different styles of cutoff proofs.  In this note we will informally consider a cutoff proof to be an inductive proof on $\mathbb{N}$, where the cutoff is the highest natual that we use in the base case.  Thus, the cutoff proof will be a proof for the consecution step in the invariant method; initiation must still be proved in the usual way.

\subsection{Initiation}
Clearly it is the case that $Init \rightarrow Safety$.

\subsection{Consecution}
As mentioned in section \ref{ssec:why}, we will use a cutoff proof to establish consecution.  We begin by establishing a key lemma:

\begin{lemma}
  Let $S(n) := \{s | s \models Safety(n)\}$.  Then $\forall n \in \mathbb{N}, S(n+1) = S(n) \cup \{(cs = \{n+1\})\}$.
\end{lemma}
\begin{proof}
  Let $n \in \mathbb{N}$ be given.  By \textit{TypeOK}, the entire state space is $\{(cs = x) | x \subseteq ProcSet\}$.  Now
  \begin{align*}
    S(n) = &\{s | s \models Safety(n)\}\\
    = &\{(cs = \emptyset), (cs = \{0\}), ..., (cs = \{n\})\}\\
  \end{align*}

  Likewise, $S(n+1) = \{(cs = \emptyset), (cs = \{0\}), ..., (cs = \{n+1\})\}$.  Hence
  \begin{align*}
    S(n+1) = &\{(cs = \emptyset), (cs = \{0\}), ..., (cs = \{n+1\})\}\\
    = &\{(cs = \emptyset), (cs = \{0\}), ..., (cs = \{n\})\} \cup \{(cs = \{n+1\})\}\\
    = &S(n) \cup \{(cs = \{n+1\})\}\\
  \end{align*}
\end{proof}

Next we present an argument the inductive cutoff proof to establish consecution.

\begin{lemma}
  $Safety$ is an inductive invariant for ToyCS, and hence is an inductive invariant for the finite instantiation of each $n \in \mathbb{N}$.  More precisely, $\forall n \in \mathbb{N}, Post(S(n)) \subseteq S(n)$.
\end{lemma}
\begin{proof}
  In the base case, let $n=0$ and then $Post(S(0)) = \emptyset \subseteq S(0)$.  Now assume that for $n \in \mathbb{N}$, $Post(S(n)) \subseteq S(n)$.  Then
  \begin{align*}
    Post(S(n+1)) = &Post(S(n) \cup (cs = \{n+1\}))\\
    = &Post(S(n) \cup \emptyset)\\
    = &Post(S(n))\\
    \subseteq &S(n)\\
    \subset &S(n+1)
  \end{align*}

  Where $S(n) \subset S(n+1)$ clearly holds due to Lemma 1.
\end{proof}

\section{Conclusion}
We have verified ToyCS using a cutoff proof during consecusion of the invariant method.  Hopefully in the future the proof techniques for a cutoff proof will converge into a more general algorithm or technique to help us verify parametric distributed protocols.

\end{document}
