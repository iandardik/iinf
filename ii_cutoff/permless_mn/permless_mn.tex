\documentclass[12pt]{article}
\usepackage{anysize}
\marginsize{1.2cm}{1.4cm}{.4cm}{1cm}

\usepackage[normalem]{ulem}
\usepackage{amsmath}
\usepackage{amsfonts}
\usepackage{hyperref}
\usepackage{amsthm}
\usepackage{url}
\usepackage{hyperref}

\theoremstyle{definition}
\newtheorem{assumption}{Assumption}
\newtheorem{lemma}{Lemma}
\newtheorem{corollary}{Corollary}
\newtheorem{theorem}{Theorem}
\newtheorem{definition}{Definition}
\newtheorem{example}{Example}

\theoremstyle{remark}
\newtheorem{remark}{Remark}

\newcommand{\msp}{\text{ }}
\newcommand{\st}{\text{ }|\text{ }}
\newcommand{\states}{\text{States}}
\newcommand{\gr}{\text{EGr}}
\newcommand{\toground}{\text{ToEGround}}

\title{M-N Without Permutations}
\author{Ian Dardik}
\date{\today}


\begin{document}

\maketitle

\section{Introduction}
Finding an inductive invariant is key for proving the correctness of a distributed protocol with respect to a safety property.  As such, a considerable amount of effort has been dedicated to finding and proving an inductive invariant for a given system.  For example, Ivy will guide a user to interactively find an inductive invariant within the confines of a decidable fragment of FOL.  In the past few years there has also been a host of research into inductive invariant synthesis for parameterized distributed protocols.  The synthesis tools that remain within the bounds of a decidable logic fragment are able to guarantee that they produce an inductive invariant, however, any tool that produces a candidate inductive invariant for a system that falls outside of a decidable fragment offers no guarantee that the candidate is indeed correct.  In this note, we assume that a candidate inductive invariant is \textit{given} and we exclusively focus on the verification step.

We have discovered a syntactic class of protocols which exhibit a \textit{cutoff} for the number of finite protocol instances which need to be verified.  We have captured this result in the M-N Theorem.

In this note we begin by introducing the Sort-Restricted to PNF Language (SRPL), the logic language that we use to encode our class of protocols.  We then introduce our encoding of protocols as a transition system in SRPL.  Next, we will prove some key lemmas before finally presenting and proving the M-N Theorem.



\section{Sort-Restricted to PNF Language}

In this section we will define SRPL as a parameterized grammar.  SRPL formulas are parameterized by a non-sorted grammar \textit{ns} as well as single sort $E$ of indistinguishable elements.

\begin{definition}
  Let $\mathcal{V}$ be a countable set of variables, $E$ be an infinitely countable sort of indistinguishable elements, and \textit{sv} be an input grammar that may not refer to $E$.  A formula in SRPL is defined by the grammar for the production rule of \textit{srpl}:
  \begin{align*}
    &arg& &::= x \qquad &\text{for any } x \in \mathcal{V}\\
    &arg\_list& &::= arg\\
    &arg\_list& &::= arg,arg\_list\\
    &Q& &::= \forall \st \exists\\
    &srpl& &::= Q \, x \in E, \, ns(arg\_list) \qquad &\text{for any } x \in \mathcal{V}\\
    &srpl& &::= Q \, x \in E, \, srpl \qquad &\text{for any } x \in \mathcal{V}\\
  \end{align*}
\end{definition}

The input grammar \textit{ns} has a single requirement--that it cannot explicitly refer to $E$--and therefore is quite general.  We now provide an example of an input grammar to illustrate a potential use case.

\begin{example}
  Let $\mathcal{S}$ be a finite set of state variables, $\mathcal{A}$ be a countable set of constants, and let $\mathcal{V}$ be a countable set of variables.  We define the grammar \textit{sample} that is parameterized on the variable symbols $x_1,...,x_n$ by the following production rules:
  \begin{align*}
    &prim(x_1,...,x_n)& &::= v \qquad &\text{for any } v \in \mathcal{S}\\
    &prim(x_1,...,x_n)& &::= y \qquad &\text{for any } y \in \mathcal{V}\\
    &prim(x_1,...,x_n)& &::= a \qquad &\text{for any } a \in \mathcal{A}\\
    &prim(x_1,...,x_n)& &::= x_i \qquad &\text{for any } 1 \leq i \leq n\\
    &prim(x_1,...,x_n)& &::= prim(x_1,...,x_n)[prim(x_1,...,x_n)]\\
    &sample(x_1,...,x_n)& &::= prim(x_1,...,x_n) = prim(x_1,...,x_n)\\
    &sample(x_1,...,x_n)& &::= \neg sample(x_1,...,x_n)\\
    &sample(x_1,...,x_n)& &::= sample(x_1,...,x_n) \land sample(x_1,...,x_n)\\
    &sample(x_1,...,x_n)& &::= \forall x \in sample(arg\_list(x_1,...,x_n)), \, sample(x_1,...,x_n) \qquad &\text{for any } x \in \mathcal{V}\\
  \end{align*}
  Notice that \textit{sample} formulas have no way to refer to the sort $E$ directly, and hence cannot quantify over $E$ nor take its cardinality.  We will use $\lor$, $\exists$, $\rightarrow$, etc. as syntactic sugar in \textit{sample} formulas, defined in the expected way.
\end{example}

\begin{definition}[Instance]
  Let $\psi$ be a SRPL formula and let $H \subseteq E$ such that $H \neq \emptyset$.  Then we define $\psi(E \mapsto H)$ by the following rules on the SRPL grammar:
  \begin{align*}
    &x(E \mapsto H)& &:= x \qquad &\text{for any } x \in \mathcal{V}\\
    &[arg,arg\_list](E \mapsto H)& &:= arg,arg\_list\\
    &[Q \, x \in E, \, ns(arg\_list)](E \mapsto H)& &:= Q \, x \in H, \, ns(arg\_list) \qquad &\text{for any } x \in \mathcal{V}\\
    &[Q \, x \in E, \, srpl](E \mapsto H)& &:=Q \, x \in H, \, [srpl(E \mapsto H)]  \qquad &\text{for any } x \in \mathcal{V}\\
  \end{align*}
  In other words, $\psi(E \mapsto H)$ is the formula $\psi$ with $E$ replaced with $H$.  We call $\psi(E \mapsto H)$ an \textit{instance} of $\psi$, and when $H$ is finite, we call $\psi(E \mapsto H)$ a \textit{finite instance} of $\psi$.
\end{definition}

\begin{definition}[Finite Instance Notation]
  We use a special shorthand for finite instaces that mirrors the notation described in \cite{goel2021symmetry}.  Let $\psi$ be a SRPL formula and $k>0$ be given.  Then $\psi(k) := \psi(E \mapsto \{e_1,...,e_k\})$ where each $e_i \in E$ is arbitrary and distinct.  We can also write $E(k) := \{e_1,...,e_k\}$ where each $e_i \in E$ is arbitrary and distinct.
\end{definition}

\begin{definition}[Valid SPRL Formula]
  Let $\psi$ be a SRPL formula.  Then $\psi$ is valid iff $\psi(E \mapsto H)$ is valid for every $H \subseteq E$.
\end{definition}

\begin{lemma}
  \label{lem:valid-all-k}
  Let $\psi$ be a SRPL formula.  Then $\psi$ is valid iff $\psi(k)$ is valid for all $k>0$.
\end{lemma}



\section{\textit{E}-Ground Formulas}

\begin{definition}[ToEGround]
  Let $\psi$ be a SRPL formula, $R \subseteq \mathcal{V}$ be the variables that occur in $\psi$ that quantify over $E$, let $H \subseteq E$ such that $H \neq \emptyset$, and let $\rho : R \to H$ be given.  Then we define $\toground(\psi,\rho)$ by the following rules on the SRPL grammar:
  \begin{align*}
    &\toground(x,\rho)& &:= \rho(x) \qquad &\text{for any } x \in R\\
    &\toground([arg,arg\_list],\rho)& &:= \toground(arg,\rho),\toground(arg\_list,\rho)\\
    &\toground([Q \, x \in E, \, ns(arg\_list)],\rho)& &:= ns(\toground(arg\_list,\rho)) \qquad &\text{for any } x \in \mathcal{V}\\
    &\toground([Q \, x \in E, \, srpl],\rho)& &:= \toground(srpl,\rho)  \qquad &\text{for any } x \in \mathcal{V}\\
  \end{align*}
\end{definition}

\begin{definition}[EGround]
  A formula $g$ is an \textit{e-ground} formula iff there exists a SRPL formula $\psi$ and a mapping $\rho$ such that $g = \toground(\psi,\rho)$.  Moreover, we call $g$ a \textit{ground instance} of $\psi$.
\end{definition}

Notice that e-ground terms are not necessarily \textit{ground terms}, that is, terms without quantifiers.  We illustrate this in the following example.

\begin{example}
  Consider the following SRPL formula with input grammar \textit{sample}:
  $$\psi := \forall x \in E, \, A[x] \rightarrow (\exists y \in B[x], y = 0)$$
  where $A \in (E \to \{true,false\})$ and $B \in (E \to \mathbb{N})$ are state variables.  Let $H = \{e_1,e_2,e_3\}$ and $\rho(x) = e_1$, then:
  $$\toground(\psi,\rho) = A[e_1] \rightarrow (\exists y \in B[e_1], y = 0)$$
  is an e-ground term.  However, it is not a ground term because it contains a quantifier.
\end{example}

\begin{definition}[EGr]
  Let $\psi$ be a SRPL formula and let $H \subseteq E$ be finite.  Then:
  $$\gr(\psi,H) := \{g \st \exists \rho, \, g = \toground(\psi,\rho)\}$$
  $\gr(\psi,H)$ is the set of all possible e-ground formulas of the finite instance $\psi(E \mapsto H)$.
\end{definition}

\begin{example}
  Recall the SRPL formula with input grammar \textit{sample} from the previous example:
  $$\psi := \forall x \in E, \, A[x] \rightarrow (\exists y \in B[x], y = 0)$$
  Let $H = \{e_1,e_2,e_3\}$, then:
  \begin{align*}
    \gr(\psi,H) = \{&A[e_1] \rightarrow (\exists y \in B[e_1], y = 0),\\
    &A[e_2] \rightarrow (\exists y \in B[e_2], y = 0),\\
    &A[e_3] \rightarrow (\exists y \in B[e_3], y = 0)\}\\
  \end{align*}
\end{example}



\section{Transition System}

We encode a protocol as a transition system $T=(I,\Delta)$ where $I$ is the initial constraint restricted to universal quantification over $E$ and $\Delta$ is the transition relation restricted to existential quantification over $E$, and both are encoded in SRPL.  We assume that an inductive invariant candidate $\Phi$ is given in SRPL, and is restricted to universal quantification over $E$.  We use the notation $T(E \mapsto H):=(I(E \mapsto H),\Delta(E \mapsto H))$ where $H \subseteq E$.

\begin{definition}[States]
  $$\states(H) := \{s \st s \text{ is a state of } T(E \mapsto H)\}$$
  In this note we consider a ``state" $s \in \states(H)$ to be a ground formula.  More specifically, $s$ is a conjunction of constraints that describe a single state in $T(E \mapsto H)$.
\end{definition}

\begin{definition}[Inductive Invariant]
  $\Phi$ is an inductive invariant iff $I \rightarrow \Phi$ and $\Phi \land \Delta \rightarrow \Phi'$ are valid formulas.
\end{definition}



\section{Lemmas}
\begin{lemma}
  \label{lem:pnf-ground}
  Let $k \in \mathbb{N}$ such that $s \in \states(k)$ and $F$ is a universally quantified formula.  Then:
  $$(s \rightarrow F(k)) \leftrightarrow (\forall f \in \gr(F,k), s \rightarrow f)$$
\end{lemma}
\begin{proof}
  Suppose that $s \rightarrow F(k)$.  For an arbitrary formula $f \in \gr(F,k)$, $F(k) \rightarrow f$ and hence we see that $s \rightarrow F(k) \land F(k) \rightarrow f$.  It follows that $s \rightarrow f$.

  Now suppose that $\forall f \in \gr(F,k), s \rightarrow f$.  Suppose, for the sake of contradiction, that $\neg(s \rightarrow F(k))$.  Then it must be the case that $s \land \neg F(k)$.  We know that $F$ is unversally quantified, so let $F(k) := \forall x_1,...,x_m \in P, \phi(x_1,...,x_m)$ where $m \geq 1$.  Then, because $\neg F(k)$ holds, it must be the case that $\exists x_1,...,x_m \in P, \neg \phi(x_1,...,x_m)$.  However, $\phi(x_1,...,x_m) \in \gr(F,k)$ which, by our original assumption, implies $\neg s$.  Hence we have both $s$ and $\neg s$ and we have reached a contradiction.
\end{proof}



\section{The M-N Theorem}

In this section, we will establish initiation and consecution in two separate lemmas using similar techniques.  The M-N Theorem follows immediately from these two lemmas.

\begin{lemma}[M-N Initiation]
  Let $m$ be the number of variables that $I$ quantifies over.  Then if $I(m) \rightarrow \Phi(m)$ is valid, $I(k) \rightarrow \Phi(k)$ is also valid for all $k>m$.
\end{lemma}

\begin{lemma}[M-N Consecution]
  Let $m$ be the number of variables that $\Phi$ quantifies over and $n$ be the number of variables that $\Delta$ quantifies over.  Then if $\Phi(m+n)$ is inductive, $\Phi(k)$ is also inductive for any $k>m+n$.
\end{lemma}
\begin{proof}
  Assume that $[\Phi\land\Delta \rightarrow \Phi'](m+n)$ is valid.  Let $k>m+n$ be given, we want to show that $[\Phi\land\Delta \rightarrow \Phi'](k)$ is also valid.  Let $H = \{e_1,...,e_k\} \subseteq E$ be an arbitrary finite instance of $E$.  Let $s \in \states(H)$ such that $s \rightarrow \Phi(E \mapsto H)$ and let $\delta \in \gr(\Delta,H)$ such that $\delta \rightarrow \Delta(E \mapsto H)$.  Then $(s \land \delta)$ is a formula that describes the states reachable from $s$ in one ``$\delta$ step", and it suffices to show that $(s \land \delta) \rightarrow \Phi'(E \mapsto H)$.  Furthermore, let $\phi' \in \gr(\Phi',H)$ be arbitrary, then, by Lemma \ref{lem:pnf-ground} and the fact that $\Phi'$ is in PNF and universally quantified, it suffices to show that $(s \land \delta) \rightarrow \phi'$.

  Let $\alpha_1,...,\alpha_i$ be the unique elements of $\{e_1,...,e_k\}$ in $(\phi \land \delta)$, then we know that $i \leq m+n$ because $\phi \in \gr(\Phi,H)$ where $\Phi$ quantifies over $m$ variables and $\delta \in \gr(\Delta,H)$ where $\Delta$ quantifies over $n$ variables.  Let $j = m+n-i$, then we can choose $\beta_1,...,\beta_j$ such that $\{\beta_1,...,\beta_j\} \subseteq (\{e_1,...,e_k\}-\{\alpha_1,...,\alpha_i\})$.  Notice that $|\{\alpha_1,...,\alpha_i,\beta_1,...,\beta_j\}|=m+n$, and hence, by our initial assumption:
  $$[\Phi\land\Delta \rightarrow \Phi'](E \mapsto \{\alpha_1,...,\alpha_i,\beta_1,...,\beta_j\})$$
  must be a valid formula.

  Now, $s \rightarrow \Phi(E \mapsto \{\alpha_1,...,\alpha_i,\beta_1,...,\beta_j\})$ because $\Phi$ is in PNF and universally quantified (need lemma).  Furthermore, $\delta \rightarrow \Delta(E \mapsto \{\alpha_1,...,\alpha_i,\beta_1,...,\beta_j\})$ because $\Delta$ is in PNF and restricted to existential quantification (need lemma).  Thus we see:
  $$(s \land \delta) \rightarrow [\Phi\land\Delta](E \mapsto \{\alpha_1,...,\alpha_i,\beta_1,...,\beta_j\}) \rightarrow \Phi'(E \mapsto \{\alpha_1,...,\alpha_i,\beta_1,...,\beta_j\}) \rightarrow \phi'$$
\end{proof}

Next we present the M-N Theorem:

\begin{theorem}[M-N]
  Suppose that $\Phi$ is in PNF with only universal quantifiers, while $\Delta$ is in PNF with only existential quantifiers.  Let $m$ be the number of variables that $\Phi$ quantifies over and $n$ be the number of variables that $\Delta$ quantifies over.  If $\Phi(m+n)$ is an inductive invariant, then $\Phi(k)$ is also an inductive invariant for any $k>m+n$.
\end{theorem}
\begin{proof}
  This follows immediately from the previous two lemmas.
\end{proof}



\bibliographystyle{plain}
\bibliography{refs}

\end{document}
