\documentclass[12pt]{article}
\usepackage{anysize}
\marginsize{1.2cm}{1.4cm}{.4cm}{1cm}

\usepackage[normalem]{ulem}
\usepackage{amsmath}
\usepackage{amsfonts}
\usepackage{hyperref}
\usepackage{amsthm}
\usepackage{url}
\usepackage{hyperref}

\theoremstyle{definition}
\newtheorem{assumption}{Assumption}
\newtheorem{lemma}{Lemma}
\newtheorem{corollary}{Corollary}
\newtheorem{theorem}{Theorem}
\newtheorem{definition}{Definition}
\newtheorem{example}{Example}

\theoremstyle{remark}
\newtheorem{remark}{Remark}

\newcommand{\msp}{\text{ }}
\newcommand{\st}{\text{ }|\text{ }}
\newcommand{\states}{\text{States}}
\newcommand{\gr}{\text{Gr}}
\newcommand{\elems}{\text{Elems}}
\newcommand{\PQF}{\text{PQF}}
\newcommand{\fininstances}{\text{FinInstances}}
\newcommand{\replace}{\text{Replace}}
\newcommand{\perm}{\genfrac{}{}{0pt}{}}

\title{M-N Without Permutations}
\author{Ian Dardik}
\date{\today}


\begin{document}

\maketitle

\section{Introduction}
In this note, we consider the verification problem of a transition system $T=(I,\Delta)$ where $I$ is the initial constraint, $\Delta$ is the transition relation, and the system is parameterized by a single sort $E=\{e_1,...\}$ of indistinguishable elements.  

To begin, we will introduce notation for the template and finite instances of a transition system.  We adopt the convention of \cite{goel2021symmetry} where $T(E)$ is the template of $T$ and $T(|E|)$ is a finite instance.  We can also refer to the template or a finite instance of a quantified formula $F$ and the sort $E$.  For example, suppose $F$ is in Prenex Normal Form (PNF) and univerally quantifies over $j$ variables, i.e. $F$ can be written as:
$$F := \forall x_1,...,x_j \in E, \phi(x_1,...,x_j)$$
where $\phi$ is a non-quantified statement whose only free variables are $x_1,...,x_j$.  Then $F(k)$ is identical to the formula $F$, except $E$ is replaced by $E(k) \subseteq E$, where $E(k)=\{e_1,...,e_k\}$, that is, $k$ distinct arbitrary elements of $E$.  Thus we see:
$$F(k) = \forall x_1,...,x_j \in E(k), \phi(x_1,...,x_j)$$

In this note, we are concerned with the specific scenario in which we are given a candidate inductive invariant $\Phi$, and the finite instances $\Phi(1),...,\Phi(k)$ have been proved to be inductive invariants for $T(1),...,T(k)$; we want to know whether $\Phi$ is an inductive invariant for $T$.  We are specifically concerned with the case in which both $\Delta$ and $\Phi$ are written in PNF and $\Phi$ is restricted to universal quantification.

Throughout this note, we will build several lemmas that lead to an interesting result: let $m$ be the number of variables that $\Phi$ quantifies over and $n$ be the number of variables that $\Delta$ quantifies over; if we suppose that $\Phi(m+n)$ is an inductive invariant for $T(m+n)$, then $\Phi(k)$ is also an inductive invariant for $T(k)$ for all $k>m+n$.  We will refer to this as the M-N Theorem in this note.  This result is useful because it reduces the verification problem on $T$ to model checking a finite number of instances $T(1),...,T(m+n)$.  Essentially, $m+n$ is a cutoff instance size for proving that our inductive invariant holds.

Note: I think it is likely that if $\Phi(m+n)$ is an inductive invariant, then it is \textit{also} the case for $\Phi(k)$ for all $k<m+n$, but I left this out of this note for the time being to focus on the $k>m+n$ case.



\section{Notation}
\begin{definition}[\textit{E(k)}]
  We let $E(k) := \{e_1,...,e_k\} \subseteq E$, where each $e_i$ is distinct and arbitrarily chosen from $E$.  In particular, it is always the case that $|E(k)|=k$.
\end{definition}

\begin{definition}[\textit{F(k)}]
  Let $F$ be a quantified formula of the form $Q_1 x_1,...,Q_m x_m \in E, f(x_1,...,x_m)$, where each $Q_i \in \{\forall,\exists\}$.  Then for any $k>0$:
  $$F(k) := Q_1 x_1,...,Q_m x_m \in E(k), f(x_1,...,x_m)$$
\end{definition}

\begin{definition}[Finite Instances]
  Let $F$ be a quantified formula of the form $Q_1 x_1,...,Q_m x_m \in E, f(x_1,...,x_m)$, where each $Q_i \in \{\forall,\exists\}$.  Then for any $k>0$:
  $$\fininstances(F,k) := \{Q_1 x_1,...,Q_m x_m \in H, f(x_1,...,x_m) \st H \subseteq E \land |H|=k\}$$
  The sort $E$ is assumed to be unbound, and hence $\fininstances(F,k)$ is an infinite set.
\end{definition}



\section{Lemmas}
\begin{lemma}
  \label{lem:pnf-ground}
  Let $k \in \mathbb{N}$ such that $s \in \states(k)$ and $F$ is a universally quantified formula.  Then:
  $$(s \models F(k)) \leftrightarrow (\forall f \in \gr(F,k), s \models f)$$
\end{lemma}
\begin{proof}
  Suppose that $s \models F(k)$.  For an arbitrary formula $f \in \gr(F,k)$, $F(k) \models f$ and hence we see that $s \rightarrow F(k) \land F(k) \rightarrow f$.  It follows that $s \models f$.

  Now suppose that $\forall f \in \gr(F,k), s \models f$.  Suppose, for the sake of contradiction, that $s \not\models F(k)$.  Then it must be the case that $s \land \neg F(k)$.  We know that $F$ is unversally quantified, so let $F(k) := \forall x_1,...,x_m \in P, \phi(x_1,...,x_m)$ where $m \geq 1$.  Then, because $\neg F(k)$ holds, it must be the case that $\exists x_1,...,x_m \in P, \neg \phi(x_1,...,x_m)$.  However, $\phi(x_1,...,x_m) \in \gr(F,k)$ which, by our original assumption, implies $\neg s$.  Hence we have both $s$ and $\neg s$ and we have reached a contradiction.
\end{proof}

\begin{lemma}
  \label{ref:fin-instances}
  Let $F$ be a quantified formula and $k>0$ be given, then:
  $$F(k) \leftrightarrow \forall f \in \fininstances(F,k), f$$
\end{lemma}
\begin{proof}
  Let $F$ be a quantified formula of the form $Q_1 x_1,...,Q_m x_m \in E, f(x_1,...,x_m)$.

  Suppose that $F(k)$ is true.  Consider arbitrary $f \in \fininstances(F,k)$.  $f$ has the form $Q_1 x_1,...,Q_m x_m \in H, f(x_1,...,x_m)$ where $H = \{h_1,...,h_k\}$ is a particular subset of $E$.  Notice that $F(k) \rightarrow f$, and hence $f$ must be true.

  Now suppose that $\forall f \in \fininstances(F,k), f$, then the result follows trivially.
\end{proof}



\section{The M-N Theorem}

In this section, we will establish initiation and consecution in two separate lemmas using similar techniques.  The M-N Theorem follows immediately from these two lemmas.

\begin{lemma}[M-N Initiation]
  Suppose that $\Phi(m)$ is an inductive invariant for $T(m)$, then $I(k) \rightarrow \Phi(k)$ for all $k>m$.
\end{lemma}

\begin{lemma}[M-N Consecution]
  Suppose that $\Phi$ and $\Delta$ are both in PNF, while $\Phi$ is restricted to universal quantification.  Let $m$ be the number of variables that $\Phi$ quantifies over and $n$ be the number of variables that $\Delta$ quantifies over.  Then if $\Phi(m+n)$ is an inductive invariant, $\Phi(k)$ is inductive for any $k>m+n$.
\end{lemma}
\begin{proof}
  Assume that $[\Phi\land\Delta \rightarrow \Phi'](m+n)$ is valid.  Let $k>m+n$ be given, we want to show that $[\Phi\land\Delta \rightarrow \Phi'](k)$ is also valid.  We will fix the sort domain as $E(k) = \{e_1,...,e_k\}$ (need to clean up notation here).  Let $s \in \states(k)$ such that $s \models \Phi(k)$ and let $\delta \in \gr(\Delta,k)$ such that $\delta \models \Delta(k)$.  Then $(s \land \delta)$ is a formula that describes the states reachable from $s$ in one ``$\delta$ step", and it suffices to show that $(s \land \delta) \models \Phi'(k)$.  Furthermore, let $\phi' \in \gr(\Phi',k)$ be arbitrary, then, by Lemma \ref{lem:pnf-ground} and the fact that $\Phi'$ is in PNF and universally quantified, it suffices to show that $(s \land \delta) \models \phi'$.

  Let $\alpha_1,...,\alpha_i$ be the unique elements of $\{e_1,...,e_k\}$ in $(\phi \land \delta)$, then we know that $i \leq m+n$ because $\phi \in \gr(\Phi,k)$ where $\Phi$ quantifies over $m$ variables and $\delta \in \gr(\Delta,k)$ where $\Delta$ quantifies over $n$ variables.  Let $j = m+n-i$, then we can choose $\beta_1,...,\beta_j$ such that $\{\beta_1,...,\beta_j\} \subseteq (\{e_1,...,e_k\}-\{\alpha_1,...,\alpha_i\})$.  It is clear that:
  $$[\Phi\land\Delta](E \mapsto \{\alpha_1,...,\alpha_i,\beta_1,...,\beta_j\}) \in \fininstances(F,m+n)$$
  Now, $s \models \Phi(E \mapsto \{\alpha_1,...,\alpha_i,\beta_1,...,\beta_j\})$ because $\Phi$ is in PNF and universally quantified (need lemma).  Furthermore, $\delta \models \Delta(E \mapsto \{\alpha_1,...,\alpha_i,\beta_1,...,\beta_j\})$ because...? (this is clear if $\Delta$ is restricted to existential quantification).  Thus we see:
  $$(s \land \delta) \models [\Phi\land\Delta](E \mapsto \{\alpha_1,...,\alpha_i,\beta_1,...,\beta_j\}) \in \fininstances(F,m+n)$$
  By our initial assumption and Lemma \ref{ref:fin-instances}, $[\Phi\land\Delta \rightarrow \Phi'](E \mapsto \{\alpha_1,...,\alpha_i,\beta_1,...,\beta_j\})$ is a valid formula.  Hence:
  $$(s \land \delta) \models \Phi'(E \mapsto \{\alpha_1,...,\alpha_i,\beta_1,...,\beta_j\}) \models \phi'$$
\end{proof}

Next we present the M-N Theorem:

\begin{theorem}[M-N]
  Suppose that $\Phi$ is in PNF with only universal quantifiers, while $\Delta$ is in PNF with only existential quantifiers.  Let $m$ be the number of variables that $\Phi$ quantifies over and $n$ be the number of variables that $\Delta$ quantifies over.  If $\Phi(m+n)$ is an inductive invariant, then $\Phi(k)$ is also an inductive invariant for any $k>m+n$.
\end{theorem}
\begin{proof}
  This follows immediately from the previous two lemmas.
\end{proof}



\bibliographystyle{plain}
\bibliography{refs}

\end{document}
