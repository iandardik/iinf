\documentclass[12pt]{article}
\usepackage{anysize}
\marginsize{1.2cm}{1.4cm}{.4cm}{1cm}

\usepackage[normalem]{ulem}
\usepackage{amsmath}
\usepackage{amsfonts}
\usepackage{hyperref}
\usepackage{amsthm}
\usepackage{url}
\usepackage{hyperref}

\theoremstyle{definition}
\newtheorem{assumption}{Assumption}
\newtheorem{lemma}{Lemma}
\newtheorem{corollary}{Corollary}
\newtheorem{theorem}{Theorem}
\newtheorem{definition}{Definition}
\newtheorem{example}{Example}

\theoremstyle{remark}
\newtheorem{remark}{Remark}

\newcommand{\msp}{\text{ }}
\newcommand{\st}{\text{ }|\text{ }}
\newcommand{\states}{\text{States}}
\newcommand{\gr}{\text{EGr}}
\newcommand{\srpl}{\text{SRPL}}
\newcommand{\SRPL}{\text{SRPL}}
\newcommand{\toground}{\text{ToEGround}}

\title{A Cutoff Rule For A Special Class Of Parameterized Distributed Protocols}
\author{Ian Dardik}
\date{\today}


\begin{document}

\maketitle

\section{Introduction}
Finding an inductive invariant is key for proving safety of a distributed protocol.  As such, a considerable amount of effort has been dedicated to aid engineers and researchers in finding and proving an inductive invariant for a given system.  Ivy, for example, will interactively guide a user towards discovering an inductive invariant, while many other tools attempt to synthesize an inductive invariant with little to no help.  Many tools operate within the confines of a decidable fragment of FOL which makes it possible to \textit{prove} that the output is, indeed, an inductive invariant.  However, tools that may accept protocols and properties outside of a decidable FOL fragment--such as IC3PO--offer no theoretical guarantees that the output is a correct inductive invariant, and may rely on heuristics instead.

In this note, we choose to focus exclusively on verifying that a candidate is an inductive invariant, assuming that a candidate is already provided.  We have discovered a syntactic class of protocols that lie outside of a decidable logic fragment, but exhibit a \textit{cutoff} for the number of finite protocol instances which need to be verified.  We have captured this result in the M-N Theorem.

We begin by introducing the Sort-Quantifiers Restricted to Prenex Normal Form Language (SRPL), the logic language that we use to encode our class of protocols.  We then introduce our encoding of protocols as a transition system in SRPL.  Next, we will prove some key lemmas before finally presenting and proving the M-N Theorem.



\section{Sort-Quantifiers Restricted to PNF Language}

In this section we will define $\SRPL(E,G)$ as a grammar parameterized by a sort $E$ and an \textit{input grammar} $G$.

\begin{definition}
  Let $\mathcal{V}$ be a countable set of variables, $E$ be an infinitely countable sort of indistinguishable elements, and $G$ be an input grammar that may not refer to $E$.  A $\SRPL(E,G)$ formula is defined by the grammar for the production rule of \textit{srpl}:
  \begin{align*}
    &arg& &::= x \qquad &\text{for any } x \in \mathcal{V}\\
    &arg\_list& &::= arg\\
    &arg\_list& &::= arg,arg\_list\\
    &Q& &::= \forall \st \exists\\
    &srpl& &::= Q \, x \in E, \, G(arg\_list) \qquad &\text{for any } x \in \mathcal{V}\\
    &srpl& &::= Q \, x \in E, \, srpl \qquad &\text{for any } x \in \mathcal{V}\\
  \end{align*}
  We will often refer to a grammar $\SRPL(E,G)$ and implicitly assume that $E$ and $G$ are either given or previously defined.
\end{definition}

We now provide an example of an input grammar to illustrate a potential use case of the SRPL grammar.

\begin{example}
  Let $\mathcal{S}$ be a finite set of state variables, $\mathcal{A}$ be a countable set of constants, and let $\mathcal{V}$ be a countable set of variables.  We define the grammar \textit{sample} that is parameterized on the variable symbols $x_1,...,x_n$ by the following production rules:
  \begin{align*}
    &prim(x_1,...,x_n)& &::= v \qquad &\text{for any } v \in \mathcal{S}\\
    &prim(x_1,...,x_n)& &::= y \qquad &\text{for any } y \in \mathcal{V}\\
    &prim(x_1,...,x_n)& &::= a \qquad &\text{for any } a \in \mathcal{A}\\
    &prim(x_1,...,x_n)& &::= x_i \qquad &\text{for any } 1 \leq i \leq n\\
    &prim(x_1,...,x_n)& &::= prim(x_1,...,x_n)[prim(x_1,...,x_n)]\\
    &sample(x_1,...,x_n)& &::= prim(x_1,...,x_n) = prim(x_1,...,x_n)\\
    &sample(x_1,...,x_n)& &::= \neg sample(x_1,...,x_n)\\
    &sample(x_1,...,x_n)& &::= sample(x_1,...,x_n) \land sample(x_1,...,x_n)\\
    &sample(x_1,...,x_n)& &::= \forall x \in sample(arg\_list(x_1,...,x_n)), \, sample(x_1,...,x_n) \qquad &\text{for any } x \in \mathcal{V}\\
  \end{align*}
  Notice that \textit{sample} formulas have no way to refer to the sort $E$ directly, and hence cannot quantify over $E$ nor take its cardinality.  We will use $\lor$, $\exists$, $\rightarrow$, etc. as syntactic sugar in \textit{sample} formulas, defined in the expected way.

  The following is an example of a $\SRPL(E,sample)$ formula:
  $$\psi := \forall x \in E, \, A[x] \rightarrow (\exists y \in B[x], y = 0)$$
  where $A \in (E \to \{true,false\})$ and $B \in (E \to \mathcal{P}(\mathbb{N}))$ are state variables, and $\mathcal{P}$ denotes the power set.
\end{example}

The sort $E$ in a SRPL grammar is assumed to be countably infinite, however, we are particularly interested in verifying SRPL formulas for arbitrary \textit{finite} sized subsets of $E$, since the sort presumably models a real world system with finite resources.  Hence, we will be primarily concerned with a \textit{finite instance} of a formula.  We formally introduce this concept below.

\begin{definition}[Instance]
  Let $\psi$ be a $\SRPL(E,G)$ formula and $H \subseteq E$ such that $H \neq \emptyset$.  Then we define $\psi(E \mapsto H)$ by the following rules on the $\SRPL(E,G)$ grammar:
  \begin{align*}
    &x(E \mapsto H)& &:= x \qquad &\text{for any } x \in \mathcal{V}\\
    &[arg,arg\_list](E \mapsto H)& &:= arg,arg\_list\\
    &[Q \, x \in E, \, G(arg\_list)](E \mapsto H)& &:= Q \, x \in H, \, G(arg\_list) \qquad &\text{for any } x \in \mathcal{V}\\
    &[Q \, x \in E, \, srpl](E \mapsto H)& &:=Q \, x \in H, \, [srpl(E \mapsto H)]  \qquad &\text{for any } x \in \mathcal{V}\\
  \end{align*}
  In other words, $\psi(E \mapsto H)$ is the formula $\psi$ with $E$ replaced with $H$.  We call $\psi(E \mapsto H)$ an \textit{instance} of $\psi$, and when $H$ is finite, we call $\psi(E \mapsto H)$ a \textit{finite instance} of $\psi$.
\end{definition}

\begin{definition}[Finite Instance Notation]
  We may use a special shorthand for finite instaces that mirrors the notation described in \cite{goel2021symmetry}.  Let $\psi$ be a $\SRPL(E,G)$ formula and $k>0$ be given.  Then $\psi(k) := \psi(E \mapsto \{e_1,...,e_k\})$ where each $e_i \in E$ is arbitrary and distinct.
\end{definition}

\begin{definition}[Subinstance]
  Let $\psi$ be a $\SRPL(E,G)$ formula and $\psi(E \mapsto H_1)$ be an instance of $\psi$.  Then for any $H_2 \subseteq H_1$ where $H_2 \neq \emptyset$, we call $\psi(E \mapsto H_2)$ a \textit{subinstance} of $\psi(E \mapsto H_1)$.
\end{definition}

We now define validity of a SRPL formula in terms of finite instances.

\begin{definition}[Valid SRPL Formula]
  Let $\psi$ be a $\SRPL(E,G)$ formula.  Then $\psi$ is valid iff every finite instance of $\psi$ is valid.
\end{definition}

\begin{lemma}
  \label{lem:valid-all-k}
  Let $\psi$ be a $\SRPL(E,G)$ formula.  Then $\psi$ is valid iff $\psi(k)$ is valid for all $k>0$.
\end{lemma}
\begin{proof}
  Notice that, for a given $k>0$, $\psi(k)$ is valid iff $\psi(E \mapsto H)$ for arbitrary $H$ such that $|H|=|\{e_1,...,e_k\}|=k$.  Thus it follows that every $\psi(k)$ is valid iff every finite instance of $\psi$ is valid; our desired result follows immediately.
\end{proof}



\section{\textit{E}-Ground Formulas}

In this section we introduce \textit{E}-ground formulas, an important tool that will be used in the proof for the M-N Theorem.  We begin by introducing the $\toground$ operator which we then use to formally define an \textit{E}-ground formula.

\begin{definition}[ToEGround]
  Let $\psi$ be a $\SRPL(E,G)$ formula.  Additionally, let $R \subseteq \mathcal{V}$ be the variables that occur in $\psi$ that quantify over $E$, and let $\rho : R \to E$ be given.  Then we define $\toground(\psi,\rho)$ by the following rules on the $\SRPL(E,G)$ grammar:
  \begin{align*}
    &\toground(x,\rho)& &:= \rho(x) \qquad &\text{for any } x \in R\\
    &\toground([arg,arg\_list],\rho)& &:= \toground(arg,\rho),\toground(arg\_list,\rho)\\
    &\toground([Q \, x \in E, \, G(arg\_list)],\rho)& &:= G(\toground(arg\_list,\rho)) \qquad &\text{for any } x \in \mathcal{V}\\
    &\toground([Q \, x \in E, \, srpl],\rho)& &:= \toground(srpl,\rho)  \qquad &\text{for any } x \in \mathcal{V}\\
  \end{align*}
  Here we assume that each quantifier for $E$ in $\psi$ gets a unique variable name.  This assumption comes without loss of generality since we can always alpha-rename duplicate quantifier variables.
\end{definition}

\begin{definition}[EGround]
  A formula $g$ is an \textit{E-ground formula} iff there exists a SRPL formula $\psi$ and a mapping $\rho$ such that $g = \toground(\psi,\rho)$.  Moreover, we call $g$ a \textit{ground instance} of $\psi$.
\end{definition}

Notice that \textit{E}-ground formulas are not necessarily vanilla ground formulas, that is, formulas without quantifiers.  We illustrate this in the following example.

\begin{example}
  Recall the following $\SRPL(E,sample)$ formula from the previous example $\psi := \forall x \in E, \, A[x] \rightarrow (\exists y \in B[x], y = 0)$.  Assume $E = \{e_1,...\}$ and let $\rho(x) = e_1$, then:
  $$\toground(\psi,\rho) = A[e_1] \rightarrow (\exists y \in B[e_1], y = 0)$$
  is an \textit{E}-ground formula.  However, it is not a ground formula because it contains a quantifier.  Notice that we can also take the $\toground$ of a finite instance:
  $$\toground(\psi(E \mapsto \{e_1,e_2,e_3\}),\rho) = A[e_1] \rightarrow (\exists y \in B[e_1], y = 0)$$
\end{example}

We next introduce the $\gr$ operator.  This operator offers a simple notation for describing the set of all possible \textit{E}-ground instances for a given SRPL formula.

\begin{definition}[EGr]
  Let $\psi$ be a $\SRPL(E,G)$ formula and let $\psi(E \mapsto H)$ be a finite instance.  Let $R \subseteq \mathcal{V}$ be the variables that occur in $\psi$ that quantify over $E$.  Then:
  $$\gr(\psi,H) := \{g \st \exists \, \rho : R \to E, \, g = \toground(\psi,\rho)\}$$
  In other words, $\gr(\psi,H)$ is the set of all possible \textit{E}-ground formulas of the finite instance $\psi(E \mapsto H)$.
\end{definition}

\begin{example}
  Let $\psi := \forall x \in E, \, A[x] \rightarrow (\exists y \in B[x], y = 0)$ from the previous example, then:
  \begin{align*}
    \gr(\psi,\{e_1,e_2,e_3\}) = \{&A[e_1] \rightarrow (\exists y \in B[e_1], y = 0),\\
    &A[e_2] \rightarrow (\exists y \in B[e_2], y = 0),\\
    &A[e_3] \rightarrow (\exists y \in B[e_3], y = 0)\}\\
  \end{align*}
\end{example}



\section{Transition System}

Let a sort $E$ be given along with a valid SRPL input grammar $G$.  We encode a protocol as a transition system $T=(I,\Delta)$ where $I$ is the initial constraint and $\Delta$ is the transition relation, both formulas encoded in $\SRPL(E,G)$.  We assume that $I$ is restricted to universal quantification over $E$ while $\Delta$ is restricted to existential quantification over $E$.  Further assume that an inductive invariant candidate $\Phi$ is given in $\SRPL(E,G)$ and is restricted to universal quantification over $E$.  We use the notation $T(E \mapsto H):=(I(E \mapsto H),\Delta(E \mapsto H))$ where $H \subseteq E$ and $H \neq \emptyset$ to denote an \textit{instance} of $T$.

For the remainder of this note we will refer to $E$, $T$, $I$, $\Delta$, and $\Phi$ as defined above.  In particular, we will no longer think of $E$ as a generic input sort to a SRPL grammar; it is now \textit{the} sort of $T$ that we specifically use as input to the SRPL grammar that is used to encode $I$, $\Delta$, and $\Phi$.

\begin{definition}[Inductive Invariant]
  Let $\psi$ be a $\SRPL(E,G)$ formula.  Then $\psi$ is an inductive invariant iff $I \rightarrow \psi$ and $\psi \land \Delta \rightarrow \psi'$ are valid formulas.
\end{definition}

\begin{definition}[States]
  Let $H$ be a nonempty, finite subset of $E$.  Then:
  $$\states(H) := \{s \st s \text{ is a state of } T(E \mapsto H)\}$$
  In this note we consider a ``state" $s \in \states(H)$ to be a ground formula.  More specifically, $s$ is a conjunction of constraints that describe a single state in $T(E \mapsto H)$.
\end{definition}

\begin{example}
  Recall the running example with state variables $A \in (E \to \{true,false\})$ and $B \in (E \to \mathcal{P}(\mathbb{N}))$.  Here are several examples of ``states":
  \begin{align*}
    &A[e_1]=true \land B[e_1]=\{1,2,9\}& &\in \states(\{e_1\})\\
    &A[e_1]=false \land B[e_1]=\{0\}& &\in \states(\{e_1\})\\
    &A[e_1]=true \land A[e_2]=true \land B[e_1]=\emptyset \land B[e_2]=\emptyset& &\in \states(\{e_1,e_2\})\\
    &A[e_1]=true \land A[e_2]=false \land B[e_1]=\{1,2,9\} \land B[e_2]=\emptyset& &\in \states(\{e_1,e_2\})\\
    &A[e_1]=false \land A[e_2]=false \land B[e_1]=\{0,...,74\} \land B[e_2]=\{0,2,4\}& &\in \states(\{e_1,e_2\})\\
  \end{align*}
  This example showcases the power of using formulas to describe states; it allows us to reason about states across different finite instances.  For example:
  \begin{align*}
    (A[e_1]=true \land& A[e_2]=false \land B[e_1]=\{1,2,9\} \land B[e_2]=\emptyset) \rightarrow\\
    &(A[e_1]=true \land B[e_1]=\{1,2,9\})\\
  \end{align*}
  Here, the first state is stronger than the second state.
\end{example}

We now show a key lemma that shows that a state that satisfies a SRPL formula restricted to universal quantification on $E$ can be described equivalently in terms of \textit{E}-ground formulas.

\begin{lemma}
  \label{lem:pnf-ground}
  Let $\psi$ be a $\SRPL(E,G)$ formula restricted to universal quantification on $E$, and $\psi(E \mapsto H)$ be a finite instance.  Let $s \in \states(H)$, then:
  $$(s \rightarrow \psi(E \mapsto H)) \leftrightarrow (\forall g \in \gr(\psi,H), s \rightarrow g)$$
\end{lemma}
\begin{proof}
  Suppose that $s \rightarrow \psi(E \mapsto H)$.  We can write $\psi(E \mapsto H) = \forall x_1 \in H,...,\forall x_m \in H, \, F_G(x_1,...,x_m)$ where $F_G$ is a formula generated by the input grammar $G$.  Then $s \rightarrow F_G(e_1,...,e_m)$ for arbitrary $e_1,...,e_m \in H$.  However $F_G(e_1,...,e_m)$ is an arbitrary formula in $\gr(\psi,H)$, which implies that $\forall g \in \gr(\psi,H), s \rightarrow g$.

  Now suppose that $\forall g \in \gr(\psi,H), s \rightarrow g$.  Further suppose, for the sake of contradiction, that $\neg(s \rightarrow \psi(E \mapsto H))$, or equivalently $s \land \neg\psi(E \mapsto H)$.  Because $\psi$ is restricted to universal quantification, we can write $\psi(E \mapsto H) = \forall x_1 \in H,...,\forall x_m \in H, \, F_G(x_1,...,x_m)$ where $F_G$ is a formula generated by the input grammar $G$.  Thus we see $\neg\psi(E \mapsto H) = \exists x_1 \in H,...,\exists x_m \in H, \, \neg F_G(x_1,...,x_m)$.  Let $e_1,...,e_m \in H$ witness the existentials, i.e. the formula $\neg F_G(e_1,...,e_m)$ holds, or equivalently, $F_G(e_1,...,e_m) \rightarrow false$.  However, $F_G(e_1,...,e_m) \in \gr(\psi,H)$ and thus, by our initial assumption, we see $s \rightarrow F_G(e_1,...,e_m) \rightarrow false$ which implies $\neg s$ which is a contradiction.
\end{proof}



\section{Lemmas For Restricting The Sort Domain}

In this section we present two lemmas on the subinstances of a SRPL formula.  We show that, in two particular cases, a SRPL formula $\psi$ will remain invariant even when we remove elements from $E$.  This invariance is key to proving the M-N Theorem because it allows us to conclude that $\psi(k) \rightarrow \psi(m+n)$ for $k>m+n$.

The first lemma, Lemma \ref{lem:forall-lower}, considers a state $s$ that implies a property $\psi$ that is restricted to universal quantification, i.e. in the form $\psi(E \mapsto H) = \forall x_1 \in H,...,\forall x_m \in H, \, F_G(x_1,...,x_m)$.  The lemma states the intuitive fact that $s$ also implies any subinstance of $\psi(E \mapsto H)$ as well.  It is worthwhile to point out that this fact holds even when $F_G$ contains quantifiers.

The second lemma, Lemma \ref{lem:exists-lower}, considers an \textit{E}-ground formula $g$ that implies a property $\psi$ such that $g \in \gr(\psi,H)$.  In this case, $\psi$ is restricted to existential quantification, i.e. it is in the form $\psi(E \mapsto H) = \exists x_1 \in H,...,\exists x_m \in H, \, F_G(x_1,...,x_m)$.  Essentially, the lemma states that if $g \rightarrow \psi(E \mapsto H)$, then $g$ also implies any subinstance of $\psi(E \mapsto H)$ whose sort domain contains the sort elements that occur in $g$.  The reason this fact holds is because the sort elements that occur in $g$ witness the existentials of $\psi$ in the formula $g \rightarrow \psi(E \mapsto H)$.  It is worthwhile to point out that this fact holds as well when $F_G$ contains quantifiers.

These two lemmas also help motivate the disallowance of the sort $E$ in SRPL input grammars.  The idea is that, even if we alter the elements of $E$, the inner formula $F_G$ does not change.  Any domains that are quantified over in $F_G$ remain constant when $E$ changes, which explains why we can allow an arbitrary grammar with quantification (that doe not refer to $E$) as an input grammar to SRPL.

\begin{lemma}
  \label{lem:forall-lower}
  Let $\psi$ be a $\SRPL(E,G)$ formula restricted to universal quantification on $E$ and let $\psi(E \mapsto H_1)$ be an instance of $\psi$.  Now let $\psi(E \mapsto H_2)$ be a subinstance of $\psi(E \mapsto H_1)$, then:
  $$(s \rightarrow \psi(E \mapsto H_1)) \rightarrow (s \rightarrow \psi(E \mapsto H_2))$$
\end{lemma}
\begin{proof}
  Suppose that $s \rightarrow \psi(E \mapsto H_1)$, it suffices to show that $s \rightarrow \psi(E \mapsto H_2)$.  We know that $\psi(E \mapsto H_2)$ is in the form $\psi = \forall x_1 \in H_2,...,\forall x_m \in H_2, \, F_G(x_1,...,x_m)$, where $F_G$ is a formula generated by the input grammar $G$.  Then $s \rightarrow \psi(E \mapsto H_2)$ holds iff $s \rightarrow F_G(e_1,...,e_m)$ holds for arbitrary $e_1 \in H_2,...,e_m \in H_2$.  However, this formula must hold by our assumptions that $H_2 \subseteq H_1$ and $s \rightarrow \psi(E \mapsto H_1)$ where $\psi$ is unversally quantified over $E$.
\end{proof}

\begin{lemma}
  \label{lem:exists-lower}
  Let $\psi$ be a $\SRPL(E,G)$ formula restricted to existential quantification on $E$ and let $\psi(E \mapsto H_1)$ be a finite instance.  Now let $g \in \gr(\psi,H_1)$ and $e_1,...,e_m$ be the elements of $H_1$ that occur in $g$.  Then for any sort $H_2 \supseteq \{e_1,...,e_m\}$:
  $$(g \rightarrow \psi(E \mapsto H_1)) \rightarrow (g \rightarrow \psi(E \mapsto H_2))$$
\end{lemma}
\begin{proof}
  Suppose that $g \rightarrow \psi(E \mapsto H_1)$, then it suffices to show that $g \rightarrow \psi(E \mapsto H_2)$.  We know that $\psi$ is of the form $\psi = \exists x_1 \in H_2,...,\exists x_m \in H_2, \, F_G(x_1,...,x_m)$, where $F_G$ is a formula generated by the input grammar $G$.  Because $g \rightarrow \psi(E \mapsto H_1)$, it must be the case that $e_1,...,e_m$ witness the existential quantifiers of $\psi(E \mapsto H_1)$.  However, $\{e_1,...,e_m\} \subseteq H_2$, and hence $g \rightarrow \psi(E \mapsto H_2)$.
\end{proof}



\section{The M-N Theorem}

In this section, we will establish initiation and consecution in two separate lemmas.  The M-N Theorem is then easily proved from these two lemmas.

\begin{lemma}[M-N Initiation]
  \label{ref:initiation}
  Let $m$ be the number of quantifiers over $E$ in $I$.  Then if $I(m) \rightarrow \Phi(m)$ is valid, $I(k) \rightarrow \Phi(k)$ is also valid for all $k>m$.
\end{lemma}
\begin{proof}
  Coming soon.
\end{proof}

\begin{lemma}[M-N Consecution]
  \label{ref:consecution}
  Let $m$ be the number of quantifiers over $E$ in $\Phi$ and $n$ be the number of quantifiers over $E$ in $\Delta$.  Then if $\Phi(m+n)$ is inductive, $\Phi(k)$ is also inductive for any $k>m+n$.
\end{lemma}
\begin{proof}
  Assume that $[\Phi\land\Delta \rightarrow \Phi'](m+n)$ is valid.  Let $k>m+n$ be given, we want to show that $[\Phi\land\Delta \rightarrow \Phi'](k)$ is also valid.  Let $H = \{e_1,...,e_k\} \subseteq E$ be an arbitrary finite instance of $E$ of size $k$.  Let $s \in \states(H)$ such that $s \rightarrow \Phi(E \mapsto H)$ and let $\delta \in \gr(\Delta,H)$ such that $\delta \rightarrow \Delta(E \mapsto H)$.  Then $(s \land \delta)$ is an \textit{E}-ground formula that describes the states reachable from $s$ in one ``$\delta$ step", and it suffices to show that $(s \land \delta) \rightarrow \Phi'(E \mapsto H)$.  Furthermore, let $\phi' \in \gr(\Phi',H)$ be arbitrary, then, by Lemma \ref{lem:pnf-ground} and the fact that $\Phi'$ is restricted to universal quantification on $E$, it suffices to show that $(s \land \delta) \rightarrow \phi'$.

  Let $\alpha_1,...,\alpha_i$ be the unique elements of $\{e_1,...,e_k\}$ that occur in $(\phi \land \delta)$, then we know that $i \leq m+n$ because $\phi \in \gr(\Phi,H)$ where $\Phi$ quantifies over $m$ variables and $\delta \in \gr(\Delta,H)$ where $\Delta$ quantifies over $n$ variables.  Let $j = m+n-i$, then we can choose $\beta_1,...,\beta_j$ such that $\{\beta_1,...,\beta_j\} \subseteq (\{e_1,...,e_k\}-\{\alpha_1,...,\alpha_i\})$ (define $\{\beta_1,...,\beta_j\}=\emptyset$ in the case where $j=0$).  Notice that $|\{\alpha_1,...,\alpha_i,\beta_1,...,\beta_j\}|=m+n$, and hence, by our initial assumption:
  $$[\Phi\land\Delta \rightarrow \Phi'](E \mapsto \{\alpha_1,...,\alpha_i,\beta_1,...,\beta_j\})$$
  must be a valid formula.

  Now, $s \rightarrow \Phi(E \mapsto \{\alpha_1,...,\alpha_i,\beta_1,...,\beta_j\})$ due to Lemma \ref{lem:forall-lower} because $\Phi$ is restricted to universal quantification on $E$.  Furthermore, $\delta \rightarrow \Delta(E \mapsto \{\alpha_1,...,\alpha_i,\beta_1,...,\beta_j\})$ by Lemma \ref{lem:exists-lower} because $\Delta$ is restricted to existential quantification on $E$.  Thus we see:
  $$(s \land \delta) \rightarrow [\Phi\land\Delta](E \mapsto \{\alpha_1,...,\alpha_i,\beta_1,...,\beta_j\}) \rightarrow \Phi'(E \mapsto \{\alpha_1,...,\alpha_i,\beta_1,...,\beta_j\}) \rightarrow \phi'$$
\end{proof}

Next we present the M-N Theorem:

\begin{theorem}[M-N]
  Let $m$ be the number of quantifiers over $E$ in $\Phi$ and $n$ be the number of quantifiers over $E$ in $\Delta$.  Then if $\Phi(m+n)$ is an inductive invariant, $\Phi(k)$ is also an inductive invariant for any $k>m+n$.
\end{theorem}
\begin{proof}
  This follows immediately from Lemma \ref{ref:initiation} and Lemma \ref{ref:consecution}.
\end{proof}

Perhaps even more important than the M-N Theorem itself, is the following corollary:

\begin{corollary}
  Let $m$ be the number of quantifiers over $E$ in $\Phi$ and $n$ be the number of quantifiers over $E$ in $\Delta$.  Then if $\Phi(k)$ is an inductive invariant for all $k \in \{1,...,m+n\}$, then $\Phi$ is an inductive invariant for $T$.
\end{corollary}
\begin{proof}
  Suppose that $\Phi(k)$ is an inductive invariant for all $k \in \{1,...,m+n\}$.  By the M-N Theorem, we know that $\Phi(k)$ is also an inductive invariant for all $k>0$.  The result then follows from Lemma \ref{lem:valid-all-k}.  (TODO: a bit more is need here)
\end{proof}



\bibliographystyle{plain}
\bibliography{refs}

\end{document}
