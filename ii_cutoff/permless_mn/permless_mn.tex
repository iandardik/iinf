\documentclass[12pt]{article}
\usepackage{anysize}
\marginsize{1.2cm}{1.4cm}{.4cm}{1cm}

\usepackage[normalem]{ulem}
\usepackage{amsmath}
\usepackage{amsfonts}
\usepackage{hyperref}
\usepackage{amsthm}
\usepackage{url}
\usepackage{hyperref}

\theoremstyle{definition}
\newtheorem{assumption}{Assumption}
\newtheorem{lemma}{Lemma}
\newtheorem{corollary}{Corollary}
\newtheorem{theorem}{Theorem}
\newtheorem{definition}{Definition}
\newtheorem{example}{Example}

\theoremstyle{remark}
\newtheorem{remark}{Remark}

\newcommand{\msp}{\text{ }}
\newcommand{\st}{\text{ }|\text{ }}
\newcommand{\states}{\text{States}}
\newcommand{\gr}{\text{Gr}}
\newcommand{\elems}{\text{Elems}}
\newcommand{\PQF}{\text{PQF}}
\newcommand{\fininstances}{\text{FinInstances}}
\newcommand{\replace}{\text{Replace}}
\newcommand{\perm}{\genfrac{}{}{0pt}{}}

\title{M-N Without Permutations}
\author{Ian Dardik}
\date{\today}


\begin{document}

\maketitle

\section{Introduction}
In the past few years there has been ample research into inductive invariant synthesis for parameterized distributed protocols.  A key desirable feature for an invariant synthesis tool is the ability to check whether the algorithm terminates with a correct inductive invariant.  For tools that remain within the bounds of a decidable logic fragment, this check is feasible.  However, any tool that produces a candidate inductive invariant for a system that falls outside of a decidable fragment offers no guarantee that the candidate is indeed correct.  In this note, we assume that a candidate inductive invariant is \textit{given} and we exclusively focus on the verification step.

We have discovered a syntactic class of protocols which exhibit a \textit{cutoff} for the number of finite protocol instances which need to be verified.  We have captured this result in the M-N Theorem.

In this note we begin by introducing the Sort-Restricted to PNF Language (SRPL), the logic language that we use to encode our class of protocols.  We then introduce our encoding of protocols as a transition system in SRPL.  Next, we will prove some key lemmas before finally presenting and proving the M-N Theorem.



\section{Sort-Restricted to PNF Language}

In this section we will define SRPL as the composition of two grammars.  SRPL formulas are parameterized by a single sort $E$ of indistinguishable elements.  We assume that $E$ is countably infinite.

\begin{definition}
  Let $\mathcal{D}$ be a countable set of domain symbols, $\mathcal{P}$ be a countable set of predicates, $\mathcal{A}$ be a countable set of constants, and $\mathcal{V}$ be a countable set of variables.  A parameterized \textit{svf} term is produced by the following grammar:

  \begin{align*}
    &arg(x_1,...,x_n)& &::= y \qquad &\text{for any } y \in \mathcal{V}\\
    &arg(x_1,...,x_n)& &::= a \qquad &\text{for any } a \in \mathcal{A}\\
    &arg(x_1,...,x_n)& &::= x_i \qquad &\text{for any } 1 \leq i \leq n\\
    &arg\_list(x_1,...,x_n)& &::= arg(x_1,...,x_n)\\
    &arg\_list(x_1,...,x_n)& &::= arg(x_1,...,x_n),arg\_list(x_1,...,x_n)\\
    &svf(x_1,...,x_n)& &::= p(arg\_list(x_1,...,x_n)) \qquad &\text{for any } p \in \mathcal{P}\\
    &svf(x_1,...,x_n)& &::= \neg svf(x_1,...,x_n)\\
    &svf(x_1,...,x_n)& &::= svf(x_1,...,x_n) \land svf(x_1,...,x_n)\\
    &Q& &::= \forall \st \exists\\
    &svf(x_1,...,x_n)& &::= Q \, x \in D(arg\_list(x_1,...,x_n)), \, svf(x_1,...,x_n) \qquad &\text{for any } x \in \mathcal{V}, D \in \mathcal{D}\\
  \end{align*}
  It is important to note that no \textit{svf} term can refer to the sort $E$ directly, and hence cannot quantify over $E$ nor take its cardinality.
\end{definition}

\begin{definition}
  Let $\mathcal{V}$ be a countable set of variables.  A formula in SRPL is defined by the grammar for the production rule of \textit{srpl}:

  \begin{align*}
    &arg& &::= x \qquad &\text{for any } x \in \mathcal{V}\\
    &arg\_list& &::= arg\\
    &arg\_list& &::= arg,arg\_list\\
    &Q& &::= \forall \st \exists\\
    &srpl& &::= Q \, x \in E, \, svf(arg\_list) \qquad &\text{for any } x \in \mathcal{V}\\
    &srpl& &::= Q \, x \in E, \, srpl \qquad &\text{for any } x \in \mathcal{V}\\
  \end{align*}
\end{definition}

SRPL is quite rich and we should provide some examples to show this.

\begin{definition}[Instance]
  Let $\psi$ be an SRPL formula and let $H \subseteq E$.  Then we define $\psi(E \mapsto H)$ by the following rules on the SRPL grammar:
  \begin{align*}
    &x(E \mapsto H)& &:= x \qquad &\text{for any } x \in \mathcal{V}\\
    &arg(E \mapsto H)& &:= arg\\
    &[arg,arg\_list](E \mapsto H)& &:= arg,arg\_list\\
    &[Q \, x \in E, \, svf(arg\_list)](E \mapsto H)& &:= Q \, x \in H, \, svf(arg\_list) \qquad &\text{for any } x \in \mathcal{V}\\
    &[Q \, x \in E, \, srpl](E \mapsto H)& &:=Q \, x \in H, \, [srpl(E \mapsto H)]  \qquad &\text{for any } x \in \mathcal{V}\\
  \end{align*}
  In other words, $\psi(E \mapsto H)$ is the formula $\psi$ with $E$ replaced with $H$.  We call $\psi(E \mapsto H)$ an \textit{instance} of $\psi$, and when $H$ is finite, we call $\psi(E \mapsto H)$ a \textit{finite instance} of $\psi$.
\end{definition}

\begin{definition}[Finite Instance Notation]
  We use a special shorthand for finite instaces that mirrors the notation described in \cite{goel2021symmetry}.  Let $\psi$ be an SRPL formula and $k>0$ be given.  Then $\psi(k) := \psi(E \mapsto \{e_1,...,e_k\})$ where each $e_i \in E$ is arbitrary and distinct.  We can also write $E(k) := \{e_1,...,e_k\}$ where each $e_i \in E$ is arbitrary and distinct.
\end{definition}

\begin{definition}[Valid SPRL Formula]
  Let $\psi$ be an SRPL formula.  Then $\psi$ is valid iff $\psi(E \mapsto H)$ is valid for every $H \subseteq E$.
\end{definition}

\begin{lemma}
  \label{lem:valid-all-k}
  Let $\psi$ be an SRPL formula.  Then $\psi$ is valid iff $\psi(k)$ is valid for all $k>0$.
\end{lemma}



\section{\textit{E}-Ground Formulas}

\begin{definition}
  Let $S$ be a sort.  Then a \textit{ground formula} is generated by the following grammar:
  \begin{align*}
    argument &::= e \hfill \text{ for any } e \in S\\
    argument\_list &::= argument\\
    argument\_list &::= argument,argument\_list\\
    ground\_formula &::= p(argument\_list) \hfill \text{ for any } n\text{-ary } p \in \mathbf{P}, n \geq 0\\
  \end{align*}
  %We will let $G := \{g \st g \text{ is generated by } gr\_formula\}$ be the universe of all ground formulas.
\end{definition}

\begin{definition}
  Let $F$ be an RSL formula and $\rho : \mathbf{V} \to E$ be a function.  Then we define $\replace(F,\rho)$ recursively
  \begin{align*}
    \replace(x,\rho) &:= \rho(x) \hfill \text{ for any } x \in \mathbf{V}\\
    \replace((argument,argument\_list),\rho) &:= \replace(argument,rho), \replace(argument\_list,\rho)\\
    \replace(p(argument\_list),\rho) &:= p(\replace(argument\_list,\rho)) \hfill \text{ for any } n\text{-ary } p \in \mathbf{R}, n \geq 0\\
  \end{align*}
\end{definition}

\begin{definition}[Ground Instance of \textit{F(k)}]
  Let $F$ be a quantified PNF formula and $k \in \mathbb{N}$.  Then $g$ is a \textit{ground instance} of $F(k)$ iff there exists a mapping $\rho : \mathbf{V} \to E(k)$ and an unquantified formula $f$ such that:
  $$g = \replace(f,\rho) \text{ and } F \in \PQF(f)$$
  In other words, $g$ is a ground formula that is identical in structure to $F$ without quantifiers, and with all variables of $F(k)$ replaced by members of $E(k)$.
\end{definition}

\begin{example}
  Consider the transition system $T$ with two state variables, $x \in (P \to \mathbb{N})$ and $y \in \mathbb{Z}$.  Let $s := (x[1]=6 \land x[2]=0 \land y=-22)$ be a state in the transition system.  Let $F := \forall p,q \in P, x[p] \neq x[q]$ and $f := (x[1] \neq x[2])$.

  Then $F(2) = \forall p,q \in P(2), x[p] \neq x[q]$.  Furthermore, $f$ is a ground instance of $F(2)$, $F(2) \rightarrow f$, $s \rightarrow F(2)$, and $s \rightarrow f$.
\end{example}

\begin{definition}[Gr]
  Let $F$ be a quantified formula and $k \in \mathbb{N}$.  Then:
  $$\gr(F,k) := \{f \st f \text{ is a ground instance of } F(k)\}$$
\end{definition}

\begin{example}
  $\gr([\forall p,q \in P, p=q],2) = \{(1=1),(1=2),(2=1),(2=2)\}$\\
  Note: we sometimes use square braces to wrap formulas when it looks better than parentheses.\\
  Notice that $\gr([\forall p,q \in P, p=q],2)$ contains elements that are false.  This indicates that the statement $[\forall p,q \in P, p=q](2)$ is not valid.
\end{example}
\begin{example}
  Let $\text{sv}$ be a state variable, then:
  $$\gr((\forall p,q \in P, p \neq q \rightarrow \text{sv[p]} \neq \text{sv[q]}),3) = \{(1 \neq 1 \rightarrow \text{sv[1]} \neq \text{sv[1]}),(1 \neq 2 \rightarrow \text{sv[1]} \neq \text{sv[2]}),...\}$$
\end{example}

\begin{definition}[Elems]
  Suppose that $F$ is a quantified formula, $k \in \mathbb{N}$, and $f \in \gr(F,k)$.  Then:
  $$\elems(f) := \{e \st e \in P(k) \land e \text{ occurs in } f\}$$

  TODO make this definition better.
\end{definition}



\section{Transition System}

We encode a protocol as a transition system $T=(I,\Delta)$ where $I$ is the initial constraint restricted to universal quantification over $E$ and $\Delta$ is the transition relation restricted to existential quantification over $E$, and both are encoded in SRPL.  We assume that an inductive invariant candidate $\Phi$ is given in SRPL, and is restricted to universal quantification over $E$.  We use the notation $T(k):=(I(k),\Delta(k))$ where $k>0$.

\begin{definition}[States]
  Let $k>0$ be given, then:
  $$\states(k) := \{s \st s \text{ is a state of } T(k)\}$$
  In this note we consider a ``state" $s \in \states(k)$ to be a ground formula.  More specifically--under a given interpretation for $T$--$s$ is a conjunction of constraints that describe a single state in $T(k)$.
\end{definition}

\begin{definition}[Inductive Invariant]
  $\Phi$ is an inductive invariant iff $\Phi \land \Delta \rightarrow \Phi'$ is valid.
\end{definition}



\section{Lemmas}
\begin{lemma}
  \label{lem:pnf-ground}
  Let $k \in \mathbb{N}$ such that $s \in \states(k)$ and $F$ is a universally quantified formula.  Then:
  $$(s \rightarrow F(k)) \leftrightarrow (\forall f \in \gr(F,k), s \rightarrow f)$$
\end{lemma}
\begin{proof}
  Suppose that $s \rightarrow F(k)$.  For an arbitrary formula $f \in \gr(F,k)$, $F(k) \rightarrow f$ and hence we see that $s \rightarrow F(k) \land F(k) \rightarrow f$.  It follows that $s \rightarrow f$.

  Now suppose that $\forall f \in \gr(F,k), s \rightarrow f$.  Suppose, for the sake of contradiction, that $\neg(s \rightarrow F(k))$.  Then it must be the case that $s \land \neg F(k)$.  We know that $F$ is unversally quantified, so let $F(k) := \forall x_1,...,x_m \in P, \phi(x_1,...,x_m)$ where $m \geq 1$.  Then, because $\neg F(k)$ holds, it must be the case that $\exists x_1,...,x_m \in P, \neg \phi(x_1,...,x_m)$.  However, $\phi(x_1,...,x_m) \in \gr(F,k)$ which, by our original assumption, implies $\neg s$.  Hence we have both $s$ and $\neg s$ and we have reached a contradiction.
\end{proof}



\section{The M-N Theorem}

In this section, we will establish initiation and consecution in two separate lemmas using similar techniques.  The M-N Theorem follows immediately from these two lemmas.

\begin{lemma}[M-N Initiation]
  Let $m$ be the number of variables that $I$ quantifies over.  Then if $I(m) \rightarrow \Phi(m)$ is valid, $I(k) \rightarrow \Phi(k)$ is also valid for all $k>m$.
\end{lemma}

\begin{lemma}[M-N Consecution]
  Let $m$ be the number of variables that $\Phi$ quantifies over and $n$ be the number of variables that $\Delta$ quantifies over.  Then if $\Phi(m+n)$ is inductive, $\Phi(k)$ is also inductive for any $k>m+n$.
\end{lemma}
\begin{proof}
  Assume that $[\Phi\land\Delta \rightarrow \Phi'](m+n)$ is valid.  Let $k>m+n$ be given, we want to show that $[\Phi\land\Delta \rightarrow \Phi'](k)$ is also valid.  Let $H = \{e_1,...,e_k\} \subseteq E$ be an arbitrary finite instance of $E$.  Let $s \in \states(H)$ such that $s \rightarrow \Phi(E \mapsto H)$ and let $\delta \in \gr(\Delta,H)$ such that $\delta \rightarrow \Delta(H)$.  Then $(s \land \delta)$ is a formula that describes the states reachable from $s$ in one ``$\delta$ step", and it suffices to show that $(s \land \delta) \rightarrow \Phi'(H)$.  Furthermore, let $\phi' \in \gr(\Phi',H)$ be arbitrary, then, by Lemma \ref{lem:pnf-ground} and the fact that $\Phi'$ is in PNF and universally quantified, it suffices to show that $(s \land \delta) \rightarrow \phi'$.

  Let $\alpha_1,...,\alpha_i$ be the unique elements of $\{e_1,...,e_k\}$ in $(\phi \land \delta)$, then we know that $i \leq m+n$ because $\phi \in \gr(\Phi,H)$ where $\Phi$ quantifies over $m$ variables and $\delta \in \gr(\Delta,H)$ where $\Delta$ quantifies over $n$ variables.  Let $j = m+n-i$, then we can choose $\beta_1,...,\beta_j$ such that $\{\beta_1,...,\beta_j\} \subseteq (\{e_1,...,e_k\}-\{\alpha_1,...,\alpha_i\})$.  Notice that $|\{\alpha_1,...,\alpha_i,\beta_1,...,\beta_j\}|=m+n$, and hence, by our initial assumption:
  $$[\Phi\land\Delta \rightarrow \Phi'](E \mapsto \{\alpha_1,...,\alpha_i,\beta_1,...,\beta_j\})$$
  must be a valid formula.

  Now, $s \rightarrow \Phi(E \mapsto \{\alpha_1,...,\alpha_i,\beta_1,...,\beta_j\})$ because $\Phi$ is in PNF and universally quantified (need lemma).  Furthermore, $\delta \rightarrow \Delta(E \mapsto \{\alpha_1,...,\alpha_i,\beta_1,...,\beta_j\})$ because $\Delta$ is in PNF and restricted to existential quantification (need lemma).  Thus we see:
  $$(s \land \delta) \rightarrow [\Phi\land\Delta](E \mapsto \{\alpha_1,...,\alpha_i,\beta_1,...,\beta_j\}) \rightarrow \Phi'(E \mapsto \{\alpha_1,...,\alpha_i,\beta_1,...,\beta_j\}) \rightarrow \phi'$$
\end{proof}

Next we present the M-N Theorem:

\begin{theorem}[M-N]
  Suppose that $\Phi$ is in PNF with only universal quantifiers, while $\Delta$ is in PNF with only existential quantifiers.  Let $m$ be the number of variables that $\Phi$ quantifies over and $n$ be the number of variables that $\Delta$ quantifies over.  If $\Phi(m+n)$ is an inductive invariant, then $\Phi(k)$ is also an inductive invariant for any $k>m+n$.
\end{theorem}
\begin{proof}
  This follows immediately from the previous two lemmas.
\end{proof}



\bibliographystyle{plain}
\bibliography{refs}

\end{document}
