\documentclass[12pt]{article}
\usepackage{anysize}
\marginsize{1.2cm}{1.4cm}{.4cm}{1cm}

\usepackage[normalem]{ulem}
\usepackage{amsmath}
\usepackage{amsfonts}
\usepackage{hyperref}

\title{IC3PO Finite Convergence Check}
\author{Ian Dardik}
\date{\today}


\begin{document}

\maketitle

\section{Introduction}
IC3PO proposed an automatic method for performing ``finite convergence checks" \cite{goel2021symmetry}.  In this document I will describe the problem, the proposed algorithm, as well as issues that I discovered.  Throughout this document I will adopt the conventions as used in \cite{goel2021symmetry}, but I have chosen different variable names for convenience and readability.

\section{Preliminaries}
Let $P$ be either a property or a transition system.  Then the \textit{template} of $P$ is $P(S_1,...,S_n)$ where $n \in \mathbb{N}$ and each $S_i$ is a sort.  A finite instance of $P$ is denoted as $P(|S_1|,...,|S_n|)$, where each $|S_i| \in \mathbb{N}$ denotes the size of sort $S_i$.  For example, if we have a property $Init$ parameterized by two sorts $S_1$ and $S_2$, then $Init(S_1,S_2)$ is the template of $Init$ while $Init(2,3)$ is a finite instance.

\section{Problem Statement}
\label{stmt}
Given a transition system $T = (Init,Trans)$ with template $T(S_1,...,S_n)$ and a property $P$ with template $(S_1,...,S_n)$, we want to determine whether $P(S_1,...,S_n)$ is an inductive invariant for $T(S_1,...,S_n)$.  $P(S_1,...,S_n)$ is said to be an inductive invariant for $T(S_1,...,S_n)$ iff $P(|S_1|,...,|S_n|)$ is an inductive invariant for $T(|S_1|,...,|S_n|)$ for all $\{|S_1|,...,|S_n|\} \in \mathbb{N}^n$.


\section{Proposed Solution}
\label{sol}
Given the problem statement outlined in the previous section, \cite{goel2021symmetry} proposes the following algorithm: Manually identify a vector $V \in \mathbb{N}^n$ which assigns \textit{basesizes} to each of the $n$ sorts.  $V$ must be chosen such that $T(v_1,...,v_n)$ exhibits non-trivial behavior; ``non-trivial behavior" is glossed over in the IC3PO paper but is intuitively sort sizes large enough to see some complex behaviors in the protocol $T$.  Then $P$ is an inductive invariant if the following checks hold for $1 \leq i \leq n$:
\begin{enumerate}
  \item $Init(v_1,..,v_i+1,..,v_n) \rightarrow P(v_1,..,v_i+1,..,v_n)$
  \item $P(v_1,..,v_i+1,..,v_n) \land Trans(v_1,..,v_i+1,..,v_n) \rightarrow P'(v_1,..,v_i+1,..,v_n)$
\end{enumerate}

\section{Counterexample}
There is a simple counterexample to this proposed solution.  We begin by defining the property $R$ and then proceed to describe the counterexample.

\subsection{The \textit{R} Property}
Suppose that $T = (Init,Trans)$ is a transition system parameterized by a single sort $S$, and has at least one reachable state outside of $Init$.  Further suppose that $Q$ is an inductive invariant for $T(S)$.  Then we define $R(m) = Q \land (|S|=m \rightarrow Init)$.  Because $T$ has at least one reachable state outside of $Init$, it is clear that $Init$ is neither inductive nor invariant.  It follows that there is no value of $m \in \mathbb{N}$ such that $R(m)$ is an inductive invariant for $T(S)$.

\subsection{\textit{R}(1) Counterexample}
The proposed solution in section \ref{sol} will incorrectly decide that $R(1)$ is an inductive invariant.  To see why, consider an arbitrary basesize vector $V = \{v\}$ where $v \in \mathbb{N}$.  Then IC3PO will check the following for finite convergence:
\begin{enumerate}
  \item $Init(v+1) \rightarrow (R(1))(v+1)$
  \item $(R(1))(v+1) \land Trans(v+1) \rightarrow (R(1))'(v+1)$
\end{enumerate}

Notice that if $v>0$, then
\begin{align*}
  &(R(1))(v+1)\\
  \iff &Q \land (v+1=1 \rightarrow Init)\\
  \iff &Q \land (false \rightarrow Init)\\
  \iff &Q\\
\end{align*}

Thus, for any basesize larger than 0, the finite convergence check reduces to checking whether $Q$ is an inductive invariant.  Because $Q$ is an inductive invariant, the algorithm will incorrectly decide that $R(1)$ is an inductive invariant.

\subsection{\textit{R}(1) Counterexample in IC3PO}
The $R(1)$ counterexample is easily realized in IC3PO with the following Ivy program:

\begin{verbatim}
  #lang ivy1.7

  type num

  relation val(X:num)

  after init {
    val(X) := false;
  }

  action t(x:num) = {
    val(x) := true;
  }

  export t

  invariant [Q] val(X) | ~val(X)
  invariant [R(1)] (forall X:num,Y. X = Y) -> (forall Z. ~val(Z))
\end{verbatim}

In this Ivy program there is a single sort $num$, and $Q$ is separated from $R(1)$ for convenience.  Notice that
\begin{align*}
  &Q\\
  \iff &\forall X: \text{val(X)} \lor \neg\text{val(X)}\\
  \iff &true\\
\end{align*}

which is clearly an inductive invariant, and
\begin{align*}
  &R(1)\\
  \iff &(\forall X,Y: X=Y) \rightarrow (\forall Z, \neg\text{val(Z)})\\
  \iff &(|num|=1) \rightarrow Init\\
\end{align*}

Running IC3PO on this program with $|num|=0$ yields a property violation as expected.  Running with $|num|=1$ yields a property violation too; this is expected because IC3PO finds the property violation while it tries to synthesize an invariant (not during the finite convergence check).  For any values of $|num|>1$ however, IC3PO incorrectly posits that the conjunction of $Q$ and $R(1)$ is an inductive invariant.  Since this example is in EPR, I was able to run ivy\_check (which does not use finite interpretations of the sorts) to produce a counterexample and verify that $Q \land R(1)$ is not an inductive invariant.

\subsection{\textit{R}(m) Counterexamples}
This counterexample generalizes to higher values of $m$, though I have not tested it on IC3PO yet (because I don't know how to test for cardinality of a sort in Ivy).  Imagine the following Ivy program:

\begin{verbatim}
  #lang ivy1.7

  type num

  relation val(X:num)

  after init {
    val(X) := false;
  }

  action t(x:num) = {
    val(x) := true;
  }

  export t

  invariant [Q] val(X) | ~val(X)
  invariant [R(100)] (|num|=100) -> (forall Z. ~val(Z))
\end{verbatim}

In this case, the conjunction of the invariants is still clearly not an inductive invariant.  However, users of IC3PO will most likely choose basesizes of small integers such as 2 or 3; checking for convergence using the method in \ref{sol} will incorrectly decide that the conjunction of the invariants \textit{is} an inductive invariant.  In this case, the user would need to choose a basesize of 99 or larger for the method to work properly.

\section{Discussion}
Since $m$ can be arbitrarily chosen to create $R(m)$, \textit{no} finite number of checks on finite instances are sufficient to prove that a a property is an inductive invariant.  However, if we restrict our checks to only properties of some special form, then a finite number of checks may be possible.

\section{Conclusion}
A counterexample to IC3PO's finite convergence detection algorithm is presented.  It is possible that restricting the universe of allowed inductive invariants could make the method (or a similar one) work, and this would be a very interesting research direction.


\bibliographystyle{plain}
\bibliography{refs}

\end{document}
